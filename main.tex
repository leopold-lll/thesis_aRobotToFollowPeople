%%%%%%%%%%%%%%%%%%%%%%%%%%%%%%%%%%%%%%%%%%%%%%%%%%%%%%%%%%%%%%%%%%%%%%
%
% Template per Elaborato di Laurea
% DISI - Dipartimento di Ingegneria e Scienza dell’Informazione
%
% update 2015-09-10
%
%%%%%%%%%%%%%%%%%%%%%%%%%%%%%%%%%%%%%%%%%%%%%%%%%%%%%%%%%%%%%%%%%%%%%%

% formato FRONTE RETRO
\documentclass[epsfig,a4paper,12pt,titlepage,openany]{book}%,twoside
\usepackage{epsfig}
\usepackage{plain}
\usepackage{setspace}
\usepackage[paperheight=29.7cm,paperwidth=21cm,outer=2.5cm,inner=2.5cm,top=2.5cm,bottom=2.5cm]{geometry} % per definizione layout
%settings for double pages: outer=2cm,inner=3cm
\usepackage{titlesec} % per formato custom dei titoli dei capitoli

% Packeage added by me:
\usepackage{hyperref}	%the reference are links and not only numbers
%\usepackage[hidelinks]{hyperref} %usethis option to remove coloured rectangles around the reference (useful for print the PDF)
%The rectagles of \Cref are automatically removed during the printing
%NB: first we need to include hyperref and only after cleverref
\usepackage{cleveref}	%allow the use of \Cref to refer to elements
\usepackage{changepage}	%allow to locally modify the margins (i.e. in the front page)
\usepackage{tcolorbox} 	%allow to create rectangle to highlight text
\usepackage{pifont}		%allow to use extra characters
\usepackage{soul}		%allow the underlined text to go to a new line
\usepackage[shortlabels]{enumitem}  %enumerate with letters

%useful for side by side images display
\usepackage{graphicx}
\usepackage[labelsep=period]{caption} %labelsep=period is used to write Figure 2.4. instead of Figure 2.4:
\usepackage{subcaption}

%new commands

%%%%%%%%%%%%%%%%%%%%%%%%%%%%%%%%%%%%%%%%%%%%%%%%%%%%%%%%%%%%%%%%%%%%%%
% 
% NB: non so perchè ma anche essendo su windows il pacchetto per windows non funziona, bisogna utilizzare quello di linux per non avere errori...
%\usepackage[latin1]{inputenc} % per Windows; 
\usepackage[utf8x]{inputenc} % per Linux (richiede il pacchetto unicode);
%\usepackage[applemac]{inputenc} % per Mac.

\singlespacing
%\usepackage[italian]{babel}

%%%%%%%%%%%%%%%%%%%%%%%%%%%%%%%%%%%%%%%%%%%%%%%%%%%%%%%%%%%%%%%%%%%%%%
%%%%%%%%%%%%%%%%%%%%%%%%%%%%%%%%%%%%%%%%%%%%%%%%%%%%%%%%%%%%%%%%%%%%%%
%% Nota
%%%%%%%%%%%%%%%%%%%%%%%%%%%%%%%%%%%%%%%%%%%%%%%%%%%%%%%%%%%%%%%%%%%%%%
%% Si ricorda che il numero massimo di facciate è 30 per tesi triennale e +inf per tesi magistrale,
%% Nel conteggio delle facciate sono incluse 
%%   indice
%%   sommario
%%   capitoli
%% Dal conteggio delle facciate sono escluse
%%   frontespizio
%%   ringraziamenti
%%   allegati    
%%	 bibliografia
%%%%%%%%%%%%%%%%%%%%%%%%%%%%%%%%%%%%%%%%%%%%%%%%%%%%%%%%%%%%%%%%%%%%%%
%%%%%%%%%%%%%%%%%%%%%%%%%%%%%%%%%%%%%%%%%%%%%%%%%%%%%%%%%%%%%%%%%%%%%%

\newcommand*\NewPage{\newpage\null\thispagestyle{empty}\newpage}

\newcommand{\li}[1]{\textit{(line {#1})}} %custom command to quote the pseudocode lines in the text.
\usepackage{listings}	%package for code into the text
\renewcommand{\lstlistingname}{Algorithm}	% change the caption   of the pseudo code from Listing to Algorithm
\Crefname{lstlisting}{Algorithm}{}			% change the reference to the pseudo code from Listing to Algorithm
% latex colors (use only the first word): https://static.javatpoint.com/tutorial/latex/images/latex-colors.png
% code vertical lines: %*$\mid$*) and %*$\lfloor$*)
\lstset{
	breakatwhitespace=false,         
	breaklines=true,     
	basicstyle=\small\ttfamily,
	frame=single, 				%border of the code           
	language=Python, 			%Indica il linguaggio predefinito da usare
	,
	numbers=left,				%Numerazione di pagina
	xleftmargin=2em,
	frame=single,
	framexleftmargin=2em
	,	
	tabsize=4,
	escapeinside={\%*}{*)},
	commentstyle=\color{olive}, 			%Indica il colore dei commenti
	keywordstyle = {\color{blue}},        	%language functions
	keywordstyle = [2]{\color{teal}},   	%custom functions 
	keywordstyle = [3]{\color{black}},    	%variables
	keywordstyle = [4]{\color{purple}},   	%class_variables
	otherkeywords= {language_functions, static, False, True, foreach}, 	%addition keywords to the normal syntax
	morekeywords = [2]{custom_functions , detectPeople, slowStartPhase, biggestAreaBB, getPosition, addPositive, addNegative, pickOneNegative, leaderTrackingPhase, checkDriftProximity, follow, grab_newFrame, elapsedTime, classify, add, len, pickClosestPosition, initialize, updateRegion, onceEveryKTimes}, 
	morekeywords = [3]{variables, position, frame, phase1_length, boxes, box, stopDetections, boxesOfleader, positive, boxOfleader, defaultPosition},
	morekeywords = [4]{class_variables, detector, knn, tracker}
}

\begin{document}

	\pagenumbering{gobble}% nessuna numerazione
	%\pagestyle{plain} 	% nessuna intestazione e pie pagina con numero al centro
	
	\pagestyle{plain}

\thispagestyle{empty}

%reduce the margin by 1cm both on left and right side (AKA make the text area wider)
\begin{adjustwidth}{-1cm}{-1cm}
	\begin{center}
		\begin{figure}[h!]
			\centerline{\psfig{file=logo_unitn.pdf, width=0.6\textwidth}}
		\end{figure}
		
		\LARGE{Department of Information Engineering\\ and Computer Science\\}
		
		\vspace{1.5 cm} 
		\Large{Master’s Degree in\\}
		\huge{\textbf{Computer Science and Technologies}}
		
		\vspace{1.0 cm} 
		\Large\textsc{final dissertation\\} 
		\vspace{0.7 cm} 
		%\begin{tcolorbox}
		%	\begin{center}
				\Huge
				Integration of multiple deep learning algorithms\\
				for real-time tracking of a person\\
				in complex scenarios\\
				%\Large{\it{Sottotitolo (titolo definitivo???)}}
		%	\end{center}
		%\end{tcolorbox}
		
		
		\vspace{1.0 cm} 
		\begin{tabular*}{\textwidth}{ c @{\extracolsep{\fill}} c @{\extracolsep{\fill}} c }
			& \Large{Professor} &\\
			& \Large{Luigi Palopoli} &\\
			\Large{Co-Supervisor} & & \Large{Co-Supervisor}\\
			\Large{Stefano Divan} & & \Large{Fabiano Zenatti}\\
			& \Large{Student} &\\
			& \Large{Stefano Leonardi} &\\
		\end{tabular*}
	
		\vspace{1.0 cm} 
		\Large{Academic year 2019/2020}
	
	
		%\begin{tabular*}{\textwidth}{ c @{\extracolsep{\fill}} c }
		%	\Large{Professor} & \Large{Student}\\
		%	\Large{Luigi Palopoli}& \Large{Stefano Leonardi}\\
		%	\\
		%	\Large{Co-Supervisor} & \Large{Co-Supervisor}\\
		%	\Large{Stefano Divan}& \Large{Fabiano Zenatti}\\
		%\end{tabular*}
	\end{center}
\end{adjustwidth}

	\clearpage
	
	\thispagestyle{empty}

\begin{center}
  {\bf \Huge Ringraziamenti}
\end{center}

\vspace{4cm}


\emph{
  ...thanks to...
}
\\
Add:\\
list of acronyms\\
list of tables - shortCaption\\
list of images - shortCaption

	%creo un pagina vuota sulla DESTRA per lasciar spazio ad eventuali dediche (e avere i contenuti tutti assieme)
	%\chapter*{} \thispagestyle{empty} \addtocounter{page}{-1}
			
	\mainmatter			% inizio numerazione pagine in numeri arabi

	% indice
	\tableofcontents
	\clearpage
	      
	% gruppo per definizone di successione capitoli senza interruzione di pagina
	\begingroup
		% nessuna interruzione di pagina tra capitoli
		% ridefinizione dei comandi di clear page
		%\renewcommand{\cleardoublepage}{} %no text in this page
		%\renewcommand{\clearpage}{} %no text until next odd page
		% redefinizione del formato del titolo del capitolo
		% da formato
		%   Capitolo X
		%   Titolo capitolo
		% a formato
		%   X   Titolo capitolo
		
		\titleformat{\chapter}
			{\normalfont\Huge\bfseries}{\thechapter}{1em}{}
		
		\titlespacing*{\chapter}{0pt}{0.59in}{0.02in}
		\titlespacing*{\section}{0pt}{0.20in}{0.02in}
		\titlespacing*{\subsection}{0pt}{0.10in}{0.02in}
		
		%%%%%%%%%%%%%%%%%%%%%%%%%%%%%%%%
		% lista dei capitoli
		%
		% \input oppure \include
		%
		\phantomsection
\addcontentsline{toc}{chapter}{\listfigurename}
\listoffigures
\clearpage

%\phantomsection
%\addcontentsline{toc}{chapter}{\listtablename}
%\listoftables
%\clearpage


\chapter*{List of Acronyms}
\addcontentsline{toc}{chapter}{List of Acronyms}
\begin{itemize}
	\item BFM (Brute Force Matcher)
	\item BRIEF (Binary Robust Independent Elementary Features)
	\item CNN (Convolutional Neural Network)
	\item CPU (Central Processing Unit)
	\item CSRT (Channel and Spatial Reliability Tracker)
	\item DCF-CSR (Discriminative Correlation Filter with Channel and Spatial Reliability)
	\item DNN (Deep Neural Network)
	\item FAST (Features from Accelerated Segment Test)
	\item FCN (Fully Convolutional Network)
	\item FFT (Fast Fourier Transformations)
	\item FLANN (Fast Library for Approximate Nearest Neighbors)
	\item FoV (Field of View)
	\item FPS (Frames Per Second)
	\item GoogLeNet (Google LeNetwork)
	\item GOTURN (Generic Object Tracking Using Regression Networks)
	\item GPU (Graphics Processing Unit)
	\item IoU (Intersection Over Union)
	\item KCF (Kernelized Correlation Filters)
	\item KNN (K-Nearest Neighbors)
	\item LIDAR (Laser Imaging Detection And Ranging)
	\item mAP (mean Average Precision)
	\item MIL (Multiple Instance Learning)
	\item MM (Motion Model)
	\item MOSSE (Minimum Output Sum of Squared Error)
	\item NMS (Non-Maximum Suppression)
	\item NN (Neural Network)
	\item ORB (Oriented FAST and Rotated BRIEF)
	\item PAF (Part Affinity Fields)
	\item PRID (Person Re-IDentification)
	\item R-CNN (Region-based Convolutional Neural Networks)
	\item ResNet (Residual Network)
	\item RGB (Red Green Blue)
	\item RPN (Region Proposal Network)
	\item SIFT (Scale-Invariant Feature Transform)
	\item SSD (Single Shot multibox Detector)
	\item SSP-ReID (Saliency-Semantic Parsing Re-IDentification)
	\item STAF (Spatio-Temporal Affinity Fields)
	\item SURF (Speeded-Up Robust Features)
	\item SVM (Support Vector Machine)
	\item TLD (Tracking-Learning-Detection)
	\item YOLO (You Only Look Once)
\end{itemize}



		\clearpage
		%%%%%%%%%%%%%%%%%%%%%%%%%%%%%%%%%%%%%%%%%%%%%%%%%%%%%%%%%%%%%%%%%%%%%%
%%%%%%%%%%%%%%%%%%%%%%%%%%%%%%%%%%%%%%%%%%%%%%%%%%%%%%%%%%%%%%%%%%%%%%
% Nota:
%%%%%%%%%%%%%%%%%%%%%%%%%%%%%%%%%%%%%%%%%%%%%%%%%%%%%%%%%%%%%%%%%%%%%%
%Sommario è un breve riassunto del lavoro svolto dove si descrive l'obiettivo, l'oggetto della tesi, le metodologie e le tecniche usate, i dati elaborati e la spiegazione delle conclusioni alle quali siete arrivati.  
%
%Il sommario dell’elaborato consiste al massimo di 3 pagine e deve contenere le seguenti informazioni:
%\begin{itemize}
%	\item contesto e motivazioni 
%	\item breve riassunto del problema affrontato
%	\item tecniche utilizzate e/o sviluppate
%	\item risultati raggiunti, sottolineando il contributo personale del laureando/a
%\end{itemize}
%%%%%%%%%%%%%%%%%%%%%%%%%%%%%%%%%%%%%%%%%%%%%%%%%%%%%%%%%%%%%%%%%%%%%%
%%%%%%%%%%%%%%%%%%%%%%%%%%%%%%%%%%%%%%%%%%%%%%%%%%%%%%%%%%%%%%%%%%%%%%
\chapter*{Summary}\label{cha:summary}
\addcontentsline{toc}{chapter}{Summary}
This final dissertation is dedicated to the master thesis research project\cite{projectSourceCode} of \textbf{Stefano Leonardi} at the \textbf{University of Trento} for the \textbf{Dolomiti Robotics group}\cite{dolomitiRobotics}. The goal of this project is to implement a software able to drives a robot into a real environment.\\
\\
The scenarios that we are dealing with is a situation in which works both robots and humans. This interaction is extremely complex because the two parts need to cooperate efficiently avoiding accidents. The robot interacts with the environment with two sensors. First, the LIDAR instrument that will guarantee the driving safeness. In addition, an RGB camera is used for this project to implement a new functionality: \textit{the follow procedure}. While a human is walking the robot should in real-time recognises him as the main subject, leader, that it must follow across time through the real environment. The procedure should manage conditions in which the leader is completely hidden or it is out of the field of view for a long period of time.\\
\\
The software that will be presented has a modular structure composed of blocks. This architecture is used in order to be adaptable to future replacements of the technologies used. The general tracking problem is solved with three modules:
\begin{itemize}
	\item \textbf{Object detection} is used to periodically localize bounding boxes of people inside a frame. We have used the two actual state of the art algorithms: SSD\cite{ssd} and YOLOv3\cite{yoloV3}.
	\item \textbf{People recognition} is used to understand if the people found correspond to the leader or not. We have ideated a method based on the KNN algorithm that uses images converted to points with the DNN image classifiers: ResNet50\cite{resNet_paper} and GoogLeNet\cite{googLeNet_paper}.
	\item \textbf{Object tracking} is periodically started on the leader location and it will track it for few frames avoiding the problems generated due to long-term sequences. The methods found to be the best are: KCF\cite{kcf}, CSRT\cite{csrt} and MOSSE\cite{mosse}. Eventually, in future improvements GOTURN\cite{goturn}.
\end{itemize}
We have tested the implemented solution with an ad hoc dataset created by manually moving the robot around to record videos.\\
The experiments show that the software is able to be executed in real-time on low powerful processing units. In fact, robots cannot carry top quality hardware due to both the high cost of the GPUs and also because of these components has a too high power consumption.\\
The known weakness of the algorithm relies on the reliability of the people recognition module. In fact false-negative predictions\footnote{A false-negative prediction occurs when the leader is classified as a random person.} will affect the KNN potentialities and cause bigger and bigger problems as long as the algorithm is executed. Despite this, the procedure is able to manage minute-long frame sequences.








		\clearpage
		\chapter{Introduction} \label{cha:introduction}
This chapter offers an overview of the project on which the thesis is based. The goal is to explain in detail how the practical problem has been approached in order to analyse the physical constraints, ideate a software method able to solve them, and how these ideas were then implemented into a working algorithm. 


\section{Physical context}
The physical component in this project is a robot. Its definition can vary a lot basing on the context in which it is used. For this project, a robot can be described as a vehicle able to move in the space. A \textbf{LIDAR (Laser Imaging Detection And Ranging)} sensor is mounted. Which allows to drive in the space avoiding physical obstacles during the movement. In addition, it is installed a computational device connected to a webcam that can record streams of images representing the space in front of the robot itself. The robots that we are working with are shown in~\Cref{fig:robots}.\\
The video camera becomes the eye of the robot itself, and the captured video stream is used as the input of the algorithm working on the computational device. This computer can be composed of a \textbf{CPU (Central Processing Unit)} or more likely it is built with a \textbf{GPU (Graphics Processing Unit)} that can speed up parallelized computation, applied on the \textbf{DNNs (Deep Neural Networks)} used as the core of the algorithm. The software does not assume one component over the other, the only variation is in the performances: a GPU computation speed can be much higher than a CPU.\\
Instead, the output of the algorithm is a position composed of X and Y pixel coordinates calculated on a single frame captured from the webcam. This location can be then elaborated and, with the use of LIDAR sensor, the robot can estimate which is the 3D position of the element tracked from the software.\\
Finally, it moves to reach that position, in order to follow the tracked subject not only into the virtual space but also into the real environment.

\begin{figure}[!h]
	\centering
	\begin{subfigure}{0.44\textwidth}
		\includegraphics[width=\linewidth]{images/introduction/robot_shelfy}
		\captionsetup{margin=0.5cm}
		\caption{The first prototype designed to test applications that will be used in an industrial environment.}
	\end{subfigure}
	\begin{subfigure}{0.44\textwidth}
		\includegraphics[width=\linewidth]{images/introduction/robot_shelfini}
		\captionsetup{margin=0.5cm}
		\caption{A small model designed to build a fleet of robots that interact with each other while moving.}
	\end{subfigure}
	\captionsetup{margin=0.5cm}
	\caption[Pictures of the robots used for this thesis.]{These are the two model of robots on top of which we have designed this thesis project. The two models share the same physical characteristics.}
	\label{fig:robots}
\end{figure}

\section{The problem}
The thesis project\cite{projectSourceCode} is based on an internship with Dolomiti Robotics\cite{dolomitiRobotics}, a company working on self-driving robots.\\
These vehicles are designed to work in an industrial environment. This scenario is populated not only by robots but also people, making the driving task even more complex to achieve.

\subsection{Robot only environment}
A completely automated environment, where humans cannot access looks to be a similar context. Instead, it is completely different because each vehicle has is its own logic that can be designed to fit the requirements of all the other robots working in that area.\\
The typical solutions to drive a vehicle in this scenario are two:
\begin{itemize}
	\item Based on a centralized decision unit that moves all the robots simultaneously around. This unit is responsible for avoiding collisions by knowing the exact position of each single moving robot.
	\item Based on fixed rules of movement that each robot has to follow. The rules do not allow collision and the automated vehicles respect them.
\end{itemize}
Both these methods work because an automated vehicle uses a deterministic decision process and does not take arbitrary choices.

\subsection{Environment shared between robots and humans}
Instead, in a shared environment, there are a lot of elements that are not controlled by a deterministic rule. The changes in the scenario are random and prediction cannot be done. There are both fixed object that may have changed position due to external interaction, and also human that walk around with no defined rules.\\
In this scenario, it is fundamental to choose an input method that can measure the area around, in order to create an autonomous moving vehicle. Therefore LIDAR has been chosen. LIDAR is a technology that measures distances around the robot in a horizontal plane. The effect is that the robot knows in each direction which is the distance from the surrounding objects. This key idea has been used from Dolomiti Robotics, to design a software able to drive robots around avoiding collision with fixed obstacles or people walking.\\
While a robot moves around, it can construct a map of the fixed objects in the environment, measured with LIDAR. Instead, the moving objects, such as other robots or people, that are recognised as not fixed elements are not stored in the map as obstacles. This reconstruction allows the vehicle to move autonomously from one position to another knowing exactly which path to follow to reach the destination.

\subsection{Purpose of the internship}
The shared environment does not offer any real human-robot exchange. The two parts only share the same spaces. The purpose of this thesis project is to create a physical interaction between the two.\\
The goal is to create a new functionality \textit{"that allows a robot to follow a person into the real environment"}. How it works:
\begin{itemize}
	\item Track/follow is the interaction of a robot and a single person (called from now on \textbf{leader}).
	\item The leader starts the \textit{"follow"} functionality standing in front of the webcam of the robot.
	\item The robot has few seconds to recognise the person inside the camera \textbf{FOV (Field Of View)} as the leader.
	\item Then the leader can freely move around in the space. 
	\item In the meanwhile the algorithm is processing the webcam stream of images recognising the position of the leader and start tracking it in the virtual space, while following it in the real one.
	\item The tracking continues for a long period, up to several minutes until this functionality is stopped.
\end{itemize}

\subsection{Technical problems}\label{sec:technicalProblems}
This "follow" functionality may be easily solved under certain conditions. However, solving the general scenario, it is a much harder task.\\
Below is listed a small collection of the principal problems that make this functionality an extremely general one, therefore hard to solve.
\begin{itemize}
	\item The tracking should be done in real-time. It is impossible to follow a person if the processing speed is too slow. A high \textbf{FPS (Frames Per Second)} rate should be respected.
	\item The robot needs to physically follow the person meaning that the webcam cannot be fixed. By consequence, also the background is not fixed and the entire captured image, subject included, might be blurred.
	\item The person can move freely around walking fast, slow, or staying.
	\item The leader is a random person, it is not known while the algorithm was designed (no parameters can be fixed in advance).
	\item While the leader is walking around there might be also other people that interfere with the algorithm.
	\item The leader can be hidden from the webcam due to moving or static elements placed between the leader and the webcam itself.
	\item The leader can exit the field of view of the webcam disappearing until the robot rotates to watch it back again.
	\item The tracking should be performed for a long period.
\end{itemize}


\section{The solution}
The problem is complex due to its generality and the necessity to cover a lot of complementary conditions. For this reason more than one solution exists. In this thesis is presented a solution based on the combination of three methods. Each one is designed to solve sub-problem compared to this one, and none of them alone can overcome the challenge of the general task.\\

\subsection{Existing technologies}\label{sec:existingTechnologies}
The three technologies are:
\begin{itemize}
	\item \textbf{Object Detection (or Localization):} given an image the object detection task aim at processing the image and recognise which objects exist there. The detection not only need to produce a list of all the classes\footnote{There are a set of types to which each element can be associated i.e. person, dog, car, bicycle, bottle and so on.} of objects visible in the image, but also recognise in which section of the frame every single element is.\\
	The output of detection is a list of: class to which the element belongs, the probability associated and the \textbf{bounding box} defined as the smallest rectangle that contains the entire element.

	\item \textbf{Object Tracking:} in this case the input is not a single image but a video stream and an initial section\footnote{A portion of the image: a rectangle.} of it. The goal is to remember this portion of the image and recognise it in all the frames after the first one. It is important to note that the tracking procedure it is not designed to follow a person, a car or other it is designed to follow a rectangle of coloured pixels, no matter what these pixels represent.

	\item \textbf{Object Recognition:} this is a comparison between several pictures. These often represent a bounding box of the object that needs to be recognised. The procedure has a database of images each one with a specific class, and the input value is another picture, called \textbf{query}, that does not exist in the database but it represents a subject known. The goal is to extract from the database all the images that have a subject that looks similar to the one represented in the query.\\
	This application is mainly used to recognise humans, often in the video surveillance context. The database is composed of the bounding box of all the people seen, i.e. in a supermarket over the last week, and when a thief is captured and it is used as a query. So, the system should return all the images containing the thief itself.
\end{itemize}

\subsection{Limitation of known technologies}
The challenges presented previously in \Cref{sec:existingTechnologies}, can solve a small part of the general problem but each one has a technical problem~\Cref{sec:technicalProblems} that cannot be solved:
\begin{itemize}
	\item Object detection is a computationally expensive task, on a powerful GPU can run in real-time but that is not the case of the robots we are working with.\\
	In addition, the detection works frame by frame and each one is independent of the previous one. So, if a person is recognised in a frame, and in the next one, there is more than a single person the algorithm does not know the relation between all of them. Meaning that a person cannot be tracked from one frame to the next one.

	\item Object tracking, according to the name, seems the task that better match the requirement of the general problem.\\
	Despite that, the tracking does not consider that the tracked subject, the leader, cannot be hidden from the webcam. The leader should always be visible into the recorded video, and that is not the case. In addition, the leader can also exit the field of view of the robot while walking around.\\
	Lastly, all the trackers are designed to follow the subject for small periods\footnote{Each tracker works on a video of few seconds.}, after a while the tracked rectangle of coloured pixel changes and the precision of the output is no more not guaranteed. This phenomenon is known as the \textbf{drift effect}, after a while the drift is so wide that the tracked cannot be trusted anymore.

	\item Object recognition due to its requirement was not designed at all to run in real-time. In fact, it is enough to run this procedure only when a query occurs, and that does not happen more than one every second.\\
	Except that, there is a more intrinsic problem with the recognition to approach the general problem. The procedure requires a query that can be the subject at the actual frame, but then it should work on a dataset composed of old frames and these are useless to solve the actual frame.\\
	In addition, this algorithm cannot be independent, because the input values are bounding boxes of people, but these regions can be computed only with an object detection algorithm. Hence this approach cannot solve the problem independently.
\end{itemize}

This explanation shows that none of the existing proposed technologies can solve the general problem in all its parts.

\subsection{Combine known methods to solve the general task}
To solve the problem and manage all the requirements it is necessary to create a combination of known methods.\\
An example of integration of methods to solve a complex task was done by Jiang et al, in their paper\cite{multi-feature-fusion-and-YOLO} that presents a fusion of \textbf{YOLO9000}\cite{yoloV2} (the second version of \textbf{YOLO}\cite{yolo}) used as object detector and \textbf{SURF}\cite{surf} used as short-term object tracker.\\
The paper illustrate an innovative approach based on two thresholds that are used to understand when the drift of the tracker is too large and it is necessary to reinitialize it. So YOLO is executed to find the tracked subject back again and after the initialization the loop can start again.\\
\\
The method presented in that paper is an integration of two class of methods. Instead in this thesis is presented an integration of three. The third method is necessary because an additional technical problem exists~(\Cref{sec:technicalProblems}). Jiang et al. work on sport video clips where athletes are always followed by the camera and never disappear out of the field of view. In addition, occlusion can exist but are very short and the tracker is often able to overcome them.\\
Instead, in our scenario we need  to manage the disappearance of the leader behind a corner for a relatively long time. So, the object recognition method was introduced to solve this condition.\\ 
These are the main steps of the entire algorithm:
\begin{enumerate}
	\item The \underline{detection} module is executed and the bounding box of the leader is found. By assumptions, in this phase, if more than one person is simultaneously found the leader is chosen according to the area of the bounding boxes. The biggest area represents the most important person hence the leader.\\
	This step is executed several time to train the people recognition module.
	\item A new detection is executed and D people are found.
	\item Ad hoc optimizations and the \underline{people recognition} module is used to choose if the leader is contained in the list of people found:
	\begin{itemize}
		\item If \textbf{yes}: the procedure starts from point 4 (tracking)
		\item If \textbf{not}: the procedure loops again from point 2 (detection again)
	\end{itemize}
	\item The \underline{tracker} module is initialized with the bounding box found.
	\item The tracker runs for the next F frames. When stopped loop back to point 2.	
\end{enumerate}

This flow shows how detection, tracking and recognition are combined together to build a complete algorithm, that can run in real-time due to the alternation of slow and fast methods and to manage all the problematic scenarios.\\
The details will follow.


\section{Prior technical knowledge}
In this section are presented some technologies that should be known in order to well understand the algorithms presented in the next chapters.

\subsection{NN (Neural Network)} \label{sec:nn}
A NN\footnote{A visual and very well done explanation on how a NN works can be found online. The 3Blue1Brown youtube channel has published videos about the mechanics of \href{https://www.youtube.com/watch?v=IHZwWFHWa-w}{gradient descent} and about the learning phase based on \href{https://www.youtube.com/watch?v=Ilg3gGewQ5U}{back-propagation}.} is a black box containing several layers (\Cref{fig:howItWorks_NNlayers}). Each layer takes as input the output values of the layer before. The input of the NN goes into the first layer and the output comes from the last one. Each layer is composed by a certain amount of neurons (\Cref{fig:howItWorks_neutron}). Each of these neurons computes a weighted summation of all the output values of the neurons in the previous layer. For each couple of neurons in consecutive layers exists a weight that multiplies the output of the first one to generate the input of the next one. The final output of a neuron is then computed by summing up a bias value and by applying an activation function such as ReLU (Rectified Linear Unit), SoftMax or Identity.

\subsubsection*{Gradient descend}
The final prediction of the NN corresponds to the label of the neuron, in the last layer, with the highest value. The overall output of the last layer is evaluated with a \textbf{cost function}. The higher the cost the worst were the prediction. A perfect prediction will have cost equal to zero. The general cost of a model is the average cost computed on the entire input training set.\\
The training of the NN is based on the minimization of the cost function. Since that the cost function is based on thousands of dimensions that represent the weights over the entire NN, the minimization cannot be analytically solved. The solution is the \textbf{gradient descend} technique. Basically, a set of input values, and all the associated outputs, produce a specific average cost. This generated value is a point of the cost function, and it has an associated gradient. The measure of the gradient tells us how to modify the weights in order to produce a lower cost on the same input training set. Hence how to improve the NN performances by descending the cost function along the steepest path to reach a local minimum\footnote{We are "only" interest into local minimum since to find the global minimum it is necessary to explore the entire cost function and this requires a huge amount of time.}.\\
Unfortunately, the generation of the gradient with analytics techniques is unfeasible. Therefore we need the \textbf{back-propagation} technique to measure it.

\subsubsection*{Back-propagation}
Essentially the gradient represents the direction of the steepest path to the local minimum of the function. Another interpretation is that the magnitude of gradient represent for each dimension how the associated weight is useful for the final output prediction. For each input of the NN the neurons in the last layer should be tuned to produce a cost equal to zero. The variation for each neuron, compared to the correct value, depends on all the parameters of its summation. Practically, the elements involved are the neuron's output of the second-last layer and the associated weights. These parameters are linked one after the other, up to the left-most layer (the first one).\\
Therefore the last layer parameters can be computed to generate the best possible score. This variation will then influence the previous layer and its weights, then the one before and so on and so forth up to the first layer. This is the key idea of the back-propagation.\\
Once plenty of input has been passed through the NN and the tuning have been performed a lot of times, the weights are finally calibrated. The result, defined as the combination of the NN architecture and its weights is the called \textbf{model}. In this thesis work, we have never built our model but we have always used pre-trained ones.

\begin{figure}[!h]
	\centering
	\begin{subfigure}{0.49\textwidth}
		\includegraphics[width=\linewidth]{images/introduction/howItWorks_NN_layers}
		\captionsetup{margin=0.5cm}
		\caption{The general structure of a NN. The input, taken from the left, is passed through the four layers that generate the NN output on the right side. The four layers are respectively composed of 3, 4, 3 and 2 neurons.}
		\label{fig:howItWorks_NNlayers}
	\end{subfigure}
	\begin{subfigure}{0.49\textwidth}
		\includegraphics[width=\linewidth]{images/introduction/howItWorks_NN_neutron}
		\caption{The functioning scheme of a neuron. The outputs of the left-neurons are multiplied with ad hoc weights and then summed up. In addition, the right-neuron will add a bias value and apply an activation function to produce its output.}
		\label{fig:howItWorks_neutron}
	\end{subfigure}
	\captionsetup{margin=0.5cm}
	\caption{Two schemes that represent the key mechanics of a NN.}
	\label{fig:howItWorks_NN}
\end{figure}

\subsection{CNN (Convolutional Neural Network)} \label{sec:cnn}
A CNN\footnote{A visual explanation of the CNN, created by the DeepLizard youtube channel, is available \href{https://www.youtube.com/watch?v=YRhxdVk_sIs}{here}.}, also called \textbf{ConvNet}, is a standard NN extended to process images.\\
The novelty of this NN is characterized by a new type of layers: the \textbf{convolutional layers}. These layers are the main components of a CNN and are stacked one after the other. Often the end of the NN is composed of few traditional layers that compress the huge output of the previous layers, to generate the overall output of the CNN. A convolutional layer is a block created to discover patterns inside an image. These patterns are initially simple like stripes and corners (a visual example is shown in~\Cref{fig:example_CNN_patterns}). Then, by combining a lot of convolutional layers the learned patterns are always more complicated. Whether the CNN is trained to recognise people these advanced schemes may represent entire objects such as hands, arms, legs, and so on and so forth.

\subsubsection*{Convolution filter}
A convolution layer is built on top of the mathematical operation of convolution.\\
This operation is based on a \textbf{filter} that is a \textbf{small squared matrix} of real numbers ($\in [0, 1]$). This matrix will slide over all the image pixel by pixel, this shifting is called \textbf{convolving}. For each shift, the filter produces an output that is the dot-product of the actual portion of the image times the filter. In~\Cref{fig:howItWorks_CNN} is shown a numeric example of how the convolution filter works.

\begin{figure}[!h]
	\centering
	\begin{subfigure}{0.29\textwidth}
		\includegraphics[width=\linewidth]{images/introduction/example_CNN_patterns}
		\caption{Example of simple patterns generated by a CNN.}
		%For more complex image:
		%The elaboration of an input image with deeper and deeper convolutional filters. The result are increasingly complicated patterns.
		\label{fig:example_CNN_patterns}
	\end{subfigure}
	\begin{subfigure}{0.7\textwidth}
		\includegraphics[width=\linewidth]{images/introduction/howItWorks_CNN}
		\captionsetup{margin=0.5cm}
		\caption{A numeric example of a convolution operation. The top-left, 3x3, section of the image is multiplied with a dot-product by the filter. The result is placed in the output matrix in the top-left corner.}
		\label{fig:howItWorks_CNN}
	\end{subfigure}
	\captionsetup{margin=0.5cm}
	\caption{The overall mechanics of a CNN.}
	\label{fig:CNN}
\end{figure}


\section{Structure of the thesis}
The next chapters are organized as follows. This section concludes the introduction (\Cref{cha:introduction}).\\
Then follow three chapters one for each main method: object detection in~\Cref{cha:detection}, object tracking in~\Cref{cha:tracking} and object recognition in~\Cref{cha:recognition}.\\
An overview of the entire algorithm and how it works together follow  in~\Cref{cha:solution}.\\
In the end, the conclusions are presented in~\Cref{cha:conclusions}.





	\clearpage
		\chapter{Object Detection} \label{cha:detection}
This chapter explains into details what is the object detection task and the methods that can solve it efficiently. Moreover an overview of other methods is given among whose there not efficient ones and others that may seem to solve the task but do not.

\section{Task definition}
Object detection, also known as \textbf{object localization}, is an evolution of the \textbf{image classification} (\Cref{fig:imgAnalysisType}). In classification, an algorithm should produce a list of all the classes of objects inside the image. Instead, the detection not only calculates which object class exists but also how many occurrences are present for each class. Then, the complex part, and the most interesting one for this thesis application, is localizing where those elements are placed inside the image. The position is not considered as a point but as a bounding box defined as the smallest rectangle that contains the entire element. An example of object detection is shown in~\Cref{fig:sampleYolo}.
\begin{figure}[!h]
	\centering
	\includegraphics[width=0.8\linewidth]{images/detection/ex1_yolo}
	\caption[Object detection applied on a sample image.]{Object detection applied on a sample image\protect\footnotemark.}
	\label{fig:sampleYolo}
\end{figure}
%the footnote placed in the capion are a little bit tricky. Look at: https://tex.stackexchange.com/questions/10181/using-footnote-in-a-figures-caption
\footnotetext{Several images and implementing ideas of this thesis come from the online blogs pyImageSearch\cite{pyImageSearch} and LearnOpenCV\cite{learnOpenCV}.}

\subsection{Similar tasks}
Object detection can be additionally improved to extract even more information from an image.\\
The main evolutions, shown in~\Cref{fig:imgAnalysisType} are:
\begin{itemize}
	\item \textbf{Semantic segmentation:} takes all the bounding boxes produced by an object detector, and for each one, it calculates the pixels that belong to the object itself and the ones that do not. By doing this each class has its own colour associated. As a result, the algorithm knows for each pixel if it belongs to one label associated with the image (semantic division) or to the background (\textbf{\textcolor{orange}{yellow}} in the image).
	\item \textbf{Instance segmentation:} is similar to semantic segmentation, but in this case, each instance of the object is considered as a new element. In fact, the three cubes in the figure have different colours associated to them.\\
	This task is solved by the \textbf{Mask-R-CNN algorithm} (\Cref{sec:mask-r-cnn}).
	\item \textbf{(Human) pose estimation:} is the most complex task among the five. Mainly applied to people, this challenge consists in the estimation of the 3D position of the body. The idea is to build up a skeleton of the person in the image and understand how its body limbs are positioned. This functionality is important to understand what a person is doing in the image.\\
	This task is solved by the \textbf{OpenPose estimation algorithm} (\Cref{sec:openpose}).
\end{itemize}
\begin{figure}[!h]
	\centering
	\includegraphics[width=0.7\linewidth]{images/detection/types-of-img-analysis}
	\caption{Similar problems respect to object localization/detection.}
	\label{fig:imgAnalysisType}
\end{figure}



\section{State of the art algorithms}
Object detection has a lot of applications both in real-time, such as in this thesis, but also in safety-critical scenarios like cars with autonomous driving. This division brings out two different metrics: precision and speed. The ideal detector is both fast and precise, however this algorithm does not exist yet. The methods can be divided into two categories.\\
The solutions mainly focus on speed: YOLO (\Cref{sec:yolo}) and SSD (\Cref{sec:ssd}).\\
Instead, the one mainly focused on precision is R-CNN (\Cref{sec:r-cnn}).


\subsection{YOLO (You Only Look Once)} \label{sec:yolo}
YOLO\cite{yolo} was initially designed in 2016. At that time it was the first object detector approach to use a single \textbf{CNN (Convolutional Neural Network)}. Redmon et al. goals were to create an extremely fast detector. An overview of the overall procedure is shown in~\Cref{fig:howItWorks_yolo}, and the architecture appears in~\Cref{fig:architecture_ssdVSyolo}.\\
The image shows a two step procedure, but these steps are solved in parallel. This is the core idea of the paper. A single CNN can be highly optimized.\\
The YOLO procedure works as follows\footnote{The original presentation of YOLO by Redmon at CVPR 2016 conference can be found \href{https://www.youtube.com/watch?v=NM6lrxy0bxs}{here}.}:
\begin{itemize}
	\item Preprocess: the image is resized to fit the standard input dimension of the CNN.
	\item Left-image: the picture it is divided into a grid of CxC cells.
	\item Top-image: each cell suggests some bounding boxes centred on it, that can match elements in the background. To each box is associated a value describing the probability that it contains one of the elements of the image.\\
	At most one detection per box can be selected as correct. This relies on the assumption that two correct bounding boxes cannot share the centre. This is both an efficient idea but also a big limitation. Too small elements, close to each other, cannot be both detected.
	\item Bottom-image: each cell has an associated probability regarding a label that represents the class that can be found in that cell if an element exists in it.\\
	I.e. the cyan cells means: "if there is something here, it will belong to class 'DOG' ".
	\item Right-image: the two partial elaborations are merged. The most likely bounding boxes are chosen and classes are associated to them according to the value for each cell in the probability map.
\end{itemize}
That was the first YOLO version, in this thesis is used the third\cite{yoloV3}. Mainly, the changes were about recognition of a wider set of classes and small implementing details to improve the overall precision of the algorithm.\\
\\
The output of the CNN is generated extremely fast and it is accurate however has a big problem. Often if two classes have similar probabilities or the shape of the element is not perfect YOLO might propose more than one bounding box for each element. That is the case of~\Cref{fig:sub_noNMS_yolo} where the truck is classified both as "truck" and as "car". The same happens to the person that has been seen twice.\\
To solve this, it is necessary to apply a new technique: Non-Maximum Suppression.

\begin{figure}[!h]
	\centering
	\includegraphics[width=0.8\linewidth]{images/detection/howItWorks_yolo}
	\captionsetup{margin=0.5cm}
	\caption[The steps of the YOLO algorithm.]{The YOLO image elaboration based on bounding box proposal and class probability map.}
	\label{fig:howItWorks_yolo}
\end{figure}

\subsubsection*{NMS (Non-Maximum Suppression)}
This technique\cite{nms} is a post-processing that works on the bounding boxes suggested, as output, from YOLO or other detectors. NMS does not consider the source image. The goal of this procedure is to refine the bounding boxes proposed and to choose which subset of them is better to fit the final image prediction. Two examples of applications are shown in~\Cref{fig:nms}.\\
The main flow of the algorithm is as follows:
\begin{itemize}
	\item The input is a list of all the boxes generated for a single image. Associated to each one there is its probability.
	\item The boxes are sorted in decreasing order according to the probability associated.
	\item Then, each box is accepted or rejected according to the \textbf{IoU (Intersection Over Union)}. That is the percentage of overlapping area with an already accepted box.
	\begin{itemize}
		\item If the IoU is above a certain threshold, meaning that the two boxes overlap too much, the one with the lowest probability is discarded.
		\item If that is not the case, the box is accepted as a new prediction.
	\end{itemize}
\end{itemize}
The input in~\Cref{fig:sub_noNMS}, is processed and only one box is accepted (\Cref{fig:sub_withNMS}) because the IoU is very high. Instead, in~\Cref{fig:sub_noNMS_yolo}, two boxes are removed respectively from two other separated boxes (\Cref{fig:sub_withNMS_yolo}) because two different subjects are involved.

\begin{figure}[!h]
	\centering
	\begin{subfigure}{.13\linewidth}
		\includegraphics[width=1\linewidth]{images/detection/img1_noNMS}
		\caption{Overlap bounding boxes.}
		\label{fig:sub_noNMS}
	\end{subfigure}
	\begin{subfigure}{.13\linewidth}
		\includegraphics[width=1\linewidth]{images/detection/img1_withNMS}
		\caption{Refined bounding box.}
		\label{fig:sub_withNMS}
	\end{subfigure}
	%\hspace{0.05\linewidth}
	\begin{subfigure}{.35\linewidth}
		\includegraphics[width=1\linewidth]{images/detection/ex2_yolo_noNMS}
		\captionsetup{margin=0.3cm}
		\caption{YOLO generated bounding boxes.}
		\label{fig:sub_noNMS_yolo}
	\end{subfigure}
	\begin{subfigure}{.35\linewidth}
		\includegraphics[width=1\linewidth]{images/detection/ex2_yolo}
		\captionsetup{margin=0.3cm}
		\caption{Apply NMS to refine the YOLO's output.}
		\label{fig:sub_withNMS_yolo}
	\end{subfigure}
	\captionsetup{margin=0.5cm}
	\caption[Examples of application of NMS post-processing.]{Two scenarios of application of Non-Maximum Suppression algorithm. First: choose which of the 6 manual generated bounding boxes, on Audrey Hepburn's face, should be considered the correct one. Second: refinement of the YOLO prediction output, by removing the "car" and "person" prediction.}
	\label{fig:nms}
\end{figure}


\subsection{SSD (Single Shot multibox Detector)} \label{sec:ssd}
The principal competitor of YOLO is SSD\cite{ssd}. Both are based on the same principle: the use of a single Convolutional Neural Network to propose bounding boxes and associate them to classes. The CNN is optimized as much as possible to improve the speed performance and eventually even the accuracy.\\
The difference relies on how the two algorithms deal with the bounding boxes suggestion. YOLO for each cell of the grid chooses a couple of options and at most one can be chosen.\\
On the other hand, SSD works as follows (\Cref{fig:howItWorks_ssd}):
\begin{itemize}
	\item The image is divided into a grid of CxC cells, called \textbf{feature map}.
	\item Each cell can propose a set of default boxes that has a size measured in cells (i.e. 3 cells high and 2 wide).
	\item The process is repeated many times varying the value of C: the \textbf{granularity of the grid}. This guarantees that the algorithm is scale-independent matching both, big and small, subjects.\\
	In~\Cref{fig:architecture_ssdVSyolo} is shown how the convolution layers blocks are matched together only at the end.
	\item All the suggestions are merged together to produce the final proposals.
	\item SSD internally performs NMS to remove unnecessary detections.
\end{itemize}

\begin{figure}[!h]
	\centering
	\includegraphics[width=0.8\linewidth]{images/detection/howItWorks_ssd}
	\captionsetup{margin=0.5cm}
	\caption[The steps of the SSD algorithm.]{The SSD image processing and how the bounding box proposal is elaborated.}
	\label{fig:howItWorks_ssd}
\end{figure}
\begin{figure}[!h]
	\centering
	\includegraphics[width=1\linewidth]{images/detection/architecture_ssdVSyolo}
	\captionsetup{margin=0.5cm}
	\caption[Comparison of the architectures of SSD and YOLO.]{A comparison of architectures between SSD and YOLO which is designed as a compact block. Instead, SSD is modular, it is divided into convolution layers of different scales, combined at the end, to make the algorithm scale-independent.}
	\label{fig:architecture_ssdVSyolo}
\end{figure}


\subsubsection*{MobileNet}
The implementation of the project does not use a traditional version of SSD, but a lighter one. This model is a combination of SSD and mobileNet\cite{mobilenet}.\\
MobileNet is a methodology that approaches Convolutional Neural Networks to transform the architecture structure to build a much lighter version of the model. The concept was first ideated to allow low power devices, such as smartphones, to run computational expensive algorithms based on CNN.\\
The principle is to replace each standard convolution (\Cref{fig:sub_architecture_mobileNet1}) with a \textbf{Depthwise separable filter}. A standard convolution works on a grid of DxD pixels and for each one produces output features of depth M. This operation can be repeated N times for each source feature.\\
The operation is broken into two other simpler convolutions:
\begin{itemize}
	\item \textbf{Depthwise convolutional filters} (\Cref{fig:sub_architecture_mobileNet2}): produces only one feature output at a time, repeated M times for each DxD grid.
	\item \textbf{Pointwise convolution filters} (\Cref{fig:sub_architecture_mobileNet3}): extends the output feature of the depthwise filter to N output features.
\end{itemize}
The original paper demonstrates how these two operations stacked in a row, can produce results close to the correct ones.\\
The computational cost determined by the number of parameters, used by depthwise and pointwise filters, can be further reduced by randomly removing a percentage of these parameters. According to the portion of parameters removed (25\%, 50\%, 75\%), the algorithm precision is affected.

\begin{figure}[!h]
	\centering
	\begin{subfigure}[t]{0.3\textwidth}
		\includegraphics[width=\linewidth]{images/detection/architecture_mobileNet1}
		\caption{Standard convolution filters}
		\label{fig:sub_architecture_mobileNet1}
	\end{subfigure}
	\hspace{0.02\linewidth}
	\begin{subfigure}[t]{0.3\textwidth}
		\includegraphics[width=\linewidth]{images/detection/architecture_mobileNet2}
		\caption{Depthwise convolutional filters}
		\label{fig:sub_architecture_mobileNet2}
	\end{subfigure}
	\hspace{0.02\linewidth}
	\begin{subfigure}[t]{0.3\textwidth}
		\includegraphics[width=\linewidth]{images/detection/architecture_mobileNet3}
		\caption{1x1 convolutional filters called Pointwise convolution filters}
		\label{fig:sub_architecture_mobileNet3}
	\end{subfigure}
	\captionsetup{margin=0.5cm}
	\caption[The schemes of convolutions introduced by mobileNet.]{The novelty of mobileNet is that it converts a traditional convolution (A), into a combination of two lighter convolutions (B-C), that produce almost the same output.}
	\label{fig:architecture_mobileNet}
\end{figure}


\subsection{R-CNN (Region-based Convolutional Neural Networks)} \label{sec:r-cnn}
\textbf{R-CNN}\cite{r-cnn} was the first invented method of the three in this section. Differently from YOLO and SSD, it is mainly focused on performing detections with high precision, despite the processing time.\\
This algorithm is a two step object detector. The workflow, shown in~\Cref{fig:howItWorks_rcnn}, works as follows:
\begin{enumerate}
	\item A region proposal algorithm is executed on the input image and it produces 2000 bounding boxes.
	\item Each of these proposals are elaborated independently. \\
	For each box, an image classifier, based on CNN, produces features from the image and then predicts which classes they might contain.\\
	Any kind of image classifier can be used for this task resulting in an algorithm that can be easily adapted with new networks.
\end{enumerate}
The main problem is that overlapping proposals are elaborated independently. The result is that feature extraction is performed on the same area of the image multiple times. These have been solved with a second version of the algorithm, called \textbf{Fast-R-CNN}\cite{fast-r-cnn}. The feature extraction for the full image is performed before the image classification that now works on an already generated image of features.\\
An incremental improvement, comes from the third version: \textbf{Faster-R-CNN}\cite{faster-r-cnn}. It performs the feature extraction as the first step and based on it the bounding boxes are proposed with \textbf{RPN (Region Proposal Network)}. The modified structure allows ad hoc optimizations to improve the low processing time.

\begin{figure}[!h]
	\centering
	\includegraphics[width=0.8\linewidth]{images/detection/howItWorks_rcnn}
	\caption{The 2-step elaboration of R-CNN model on a sample image.}
	\label{fig:howItWorks_rcnn}
\end{figure}



\section{Other famous algorithms}
\subsection{Mask R-CNN} \label{sec:mask-r-cnn}
A variation of R-CNN that aims to solve the instance segmentation task (\Cref{fig:imgAnalysisType}) is Mask R-CNN\cite{mask-r-cnn}.\\
Mask R-CNN is built on top of two technologies:
\begin{itemize}
	\item Faster R-CNN used as an object detector.
	\item \textbf{FCN (Fully Convolutional Network)}\cite{fcn} that performs semantic segmentation.
\end{itemize}
For each bounding box, it is known the class, then FCN computes the segmentation of that class. All the shapes of elements in the image are then merged together to build the instance segmentation result. An application of this algorithm is shown in~\Cref{fig:ex2_maskRCNN}.

\begin{figure}[!h]
	\centering
	\includegraphics[width=0.7\linewidth]{images/detection/ex2_maskRCNN}
	\caption{Instance segmentation of an image using mask R-CNN.}
	\label{fig:ex2_maskRCNN}
\end{figure}


\subsection{Open Pose} \label{sec:openpose}
Firstly designed in 2017 OpenPose\cite{openpose-PAF} aims at processing an image and recognising the position of the people in it. The position is the skeleton of a person, it is the interconnection of limbs that link 15 points on the human body.\\
This algorithm opens al lot of possibilities because until then estimating the body position was achieved with 3D cameras, extremely expensive hardware that now can be easily substituted. Recently, OpenPose has been improved in terms of speed, to process frames in a video with \textbf{STAF (Spatio-Temporal Affinity Field)}\cite{openpose-STAF}, and it is also been extended to understand the 3D human position.\\
An overview of the overall procedure of the algorithm is shown in~\Cref{fig:howItWorks_openpose}, instead some output examples are shown in~\Cref{fig:ex_openpose}.\\
The OpenPose procedure works as follows\footnote{The original demo of OpenPose by Zhe Cao at CVPR 2017 conference can be found \href{https://www.youtube.com/watch?v=pW6nZXeWlGM}{here}.}:
\begin{enumerate}[a)]
	\item Preprocessing: the input image is reshaped to match the requirements of the two-branch CNN.
	\item Part Confidence Maps: the first branch of the CNN process the image to extract the location of the 15 body parts.\\
	Each body part (i.e. left shoulder, left knee, right wrist...) is detected by an ad hoc filter. These filters do not process one person at a time but the entire image simultaneously. The result is that a filter recognises all the visible right elbows in the image. This is extremely important because, by doing this, the algorithm process speed is independent respect to the number of people in the image.
	\item Part affinity fields: the second branch of the CNN process the image to recognise the limbs that can connect all the body points found so far.
	\item Bipartite matching: has the goal to match all the elements found to reconstruct the skeleton of the entire person.\\
	This reconstruction is a greedy approach mainly based on geometry that wants to minimize the distance of two parts that must be connected.
	\item Parsing results: the output image is assembled with the full-body poses for all people in the image.
\end{enumerate}
It is important to note that OpenPose is a bottom-up approach. The algorithm has no knowledge about the positioning of people in the image, it only tries to reconstruct the skeleton from small sections of the body.

\begin{figure}[!h]
	\centering
	\includegraphics[width=0.8\linewidth]{images/detection/howItWorks_openpose_2line}
	\captionsetup{margin=0.5cm}
	\caption[The steps of the OpenPose algorithm.]{The elaboration of the OpenPose algorithm to recognise the skeleton of the two dancers in the image.}
	\label{fig:howItWorks_openpose}
\end{figure}
\begin{figure}[!h]
	\centering
	\includegraphics[width=1\linewidth]{images/detection/ex_openpose_2line}
	\captionsetup{margin=0.5cm}
	\caption[Examples of application of OpenPose.]{Some examples, taken from the original paper, on how OpenPose works.}
	\label{fig:ex_openpose}
\end{figure}



\section{Overview of the algorithms}
The algorithms presented in this chapter represent the state of the art methods for their field of application.\\
All those methods can be used to achieve the long-term tracking that is the goal of this thesis, but the different information generated should be used to solve the problem with different approaches. We have chosen to use the methods that perform only the detection task. This results on one hand into general information as output, and on the other hand into a very high processing speed that is fundamental for real-time application.

\subsubsection*{Performances comparison}
A comparison of the speed, measured as \textbf{FPS (Frames Per Second)}, and of the precision, measured as \textbf{mAP (mean Average Precision)}, is shown in~\Cref{tab:detectionPerformances}. The data are based on the Pascal VOC 2007 dataset\cite{pascal-voc-2007} and come from multiple papers\cite{yolo}\cite{yoloV2}.\\
The measures do not show YOLOv3 because compared to the other methods it is more recent and a fair comparison does not exist. Instead, for Mask R-CNN and OpenPose, only the frame rate is shown because the mAP can be computed. However it is completely irrelevant respect to the other presented in the table. These two algorithms perform different tasks hence the precision of the result cannot be compared.\\
By looking at the data, it clearly appears that the single-stage algorithms (SSD and YOLO) are much faster respect to the two-stage methods (R-CNN), in fact, they run around 5-10 times faster than R-CNN. Instead, the precision of the three detectors is almost the same. For these reasons, we have chosen to use in this project the last version of YOLO and a lighter version of SSD: mobileNet-SSD. This idea pays a few percentage points in term of mAP but implements the CNN with fewer parameters, results in a low power consumption method. This aspect is important because the robots do not mount top quality hardware, therefore the light version of SSD can be executed more easily.

\subsubsection*{Output visualization}
To visually show the potentialities of these algorithms we have applied all of them on the same picture. This elaboration is presented in~\Cref{fig:ex_detectionAlgorithms}. Independently from the task that they solve, it appears evident that all of them recognise the 5 people in the foreground, but there are some differences:
\begin{itemize}
	\item SSD has troubles with the player on the left. In fact, the percentage associated with him is only a 32\%.
	\item YOLO is able to detect even a sixth person in the background that none of the others have seen.
	\item YOLO is trained to recognise a wide variety of objects respect to the other detectors, in fact, it is able to recognise even the "sports ball".\\
	Even SSD is adaptable and can be trained to recognise the ball. However the big advantage of the second version of YOLO (also called \textbf{YOLO9000}), is that it was integrated with the \textbf{Wordnet graph}\cite{wordnet} to be scalable in terms of the number of classes recognised. The result is a detector that is able to recognise up to 9000 different classes.
	\item Mask R-CNN is, in this example, the most accurate algorithm. It recognises four people with a precision of 99\% and the last one with 92\%. 
	\item OpenPose, by estimating the body parts, can even understand the orientation of the people in the soccer field. This big advantage can be used to understand where these people will move in the frame after this one.
\end{itemize}
Considering these reasons, it is evident that discarding OpenPose and Mask R-CNN, in favour of SSD and YOLO, is only a choice for this project. All these algorithms have potentiality that can be used to create a good object tracker.

\begin{table}[!h]
	\centering
	\begin{tabular}{|c|c|c|c|c|c|c|c|c|}
		\hline
		Pascal VOC 2007 & \multicolumn{2}{c|}{YOLO} &     & \multicolumn{4}{c|}{R-CNN}   &          \\ \hline
		Algorithm       & v1          & v2          & SSD & R-CNN & Fast & Faster & Mask & OpenPose \\ \hline
		FPS             & 45          & \textbf{67}          & 46  & 0.05  & 0.5  & 7      & 7    & 10       \\ \hline
		mAP             & 66          & \underline{\textbf{76}}          & \underline{74}  & 53    & 70   & \underline{73}     & X    & X        \\ \hline
	\end{tabular}
	\captionsetup{margin=0.5cm}
	\caption[Comparison of mAP and FPS for the object detector algorithms.]{Overview of the performances of the algorithms presented. The evaluation is based on the mAP and the processing speed. In bold the best scores and underlined the mAP of the three states of the art detectors.}
	\label{tab:detectionPerformances}
\end{table}
\begin{figure}[!h]
	\centering
	\begin{subfigure}{.48\linewidth}
		\includegraphics[width=\linewidth]{images/detection/ex3_yolo}
		\caption{YOLOv2: detection}
	\end{subfigure}
	\hspace{0.01\linewidth}
	\begin{subfigure}{.48\linewidth}
		\includegraphics[width=\linewidth]{images/detection/ex3_ssd}
		\caption{SSD: detection}
	\end{subfigure}
	\\
	\begin{subfigure}{.48\linewidth}
		\includegraphics[width=\linewidth]{images/detection/ex3_mask-rcnn}
		\caption{Mask R-CNN: instance segmentation}
	\end{subfigure}
	\hspace{0.01\linewidth}
	\begin{subfigure}{.48\linewidth}
		\includegraphics[width=\linewidth]{images/detection/ex3_openpose}
		\caption{OpenPose: human pose estimation}
	\end{subfigure}
	\captionsetup{margin=0.5cm}
	\caption[YOLO, SSD, Mask R-CNN and OpenPose applied on the same image.]{An example of application of the four main algorithms, based on the same sample image.}
	\label{fig:ex_detectionAlgorithms}
\end{figure}









		\clearpage
		\input{3-tracking}		\clearpage
		\chapter{People Recognition} \label{cha:recognition}
The goal of this chapter is to show which types of techniques could be used to understand, given two images of a human, if the represented person is the same or not.

\section{Problem definition}
The \textbf{people recognition module}, also known as the \textbf{people identifier module}, has the role of connecting, in a consistent way, the two remaining modules: the detector and the tracker. The tracker works for the majority of the time, following the \textbf{main subject}, which is called \textbf{leader}. When it fails or after a certain amout of time, the detector locates all the people in the incoming frame. Finally, the recognizer has to answer a simple question:
\begin{tcolorbox}
	\begin{center}
		\underline{\textit{Is this person the leader?}}\\
		Given an image cropped on the bounding box of a person and a set of other images with the same characteristics and containing various people labelled with names the recogniser has to chose which is the name of the unknown person.
	\end{center}
\end{tcolorbox}
In our specific scenario, we are interested in understanding whether a bounding box contains a representation of the leader or not. For this reason the label can be considered with only two values: "\textit{positive}" or "\textit{negative}", "\textit{leader}" or "\textit{other}".\\

\subsection{Video surveillance application}
This challenge is extremely important in the video surveillance field. In fact, one of the most common application is to reconstruct where a person was seen during a certain time slot. The video surveillance application is a little bit different from the one in which we are interested:
\begin{tcolorbox}
	\begin{center}
		Given a dataset of images of people, and a query containing an unknown image of a person all the images that contain the same person of the query are extracted from the dataset.
	\end{center}
\end{tcolorbox}
From a high-level point of view, the two problems might look different but are essentially the same. The key idea is to look for all the images that might contain the person of the query. Then, according to the task, it is necessary to retrieve this list of people or use them and all the associated information to understand who is the person of the query.\\
Another difference comes from the acquisition of images. In our scenario, the webcam is mounted on the robot that is moving around. In video surveillance the cameras are often static but several of them can simultaneously collect frames. We took inspiration from both single-camera scenarios\cite{reID_withSSN} and multi-camera systems\cite{multiFeatures_reID}.


\section{Explored methods with bad performances}
This section is focused on some unsuccessful technique was explored during the development of the general structure for the project.

\subsection{Offline algorithms}
An offline algorithm (explained in~\Cref{sec:goturn}) is a great choice for training methods able to judge situations quickly. In this case, we have considered an \textbf{SVM (Support Vector Machine) classifier}.\\
This method is a supervised learning model based on a linear separator. The essential idea is to find the line\footnote{A line in 2D problems, that is converted into a hyperplane in N-dimensional scenarios.} that divides the dataset into two sections, where respectively the two classes of interest lie (a visual representation is given in~\Cref{fig:howItWorks_svm}). The method could work if each image of a person is converted into a point, as explained in~\Cref{sec:classifiers}.\\
However this approach has an essential problem. Offline methods require prior knowledge of the context of application to train ad hoc models. In our case, the element to know is "\textit{Who is the leader?}". Based on this information the dataset can be split into two classes, the images representing the leader and the other ones, then the SVM classifier can be trained. The result is a model that always recognises one person, the same person all the times.\\
Unluckily, this prior knowledge is not given. The algorithm should recognise all types of people after seeing them for a few seconds. This is the context where an online algorithm can be wisely used.
\begin{figure}
	\centering
	\begin{minipage}{.59\textwidth}
		\centering
		\includegraphics[width=1\linewidth]{images/recognition/howItWorks_svm}
		\captionof{figure}[Graphical representation of SVM.]{The overall mechanics of SVM. The \textbf{\textcolor{red}{red}} and \textbf{\textcolor{blue}{blue}} classes are divided by the hyperplane that maximizes the margin length, defined as the shortest distance from the closest point on both sides.}
		\label{fig:howItWorks_svm}
	\end{minipage}
	\begin{minipage}{.39\textwidth}
		\centering
		\includegraphics[width=1\linewidth]{images/recognition/kpMatch_regions}
		\captionsetup{margin=0.5cm}
		\captionof{figure}[Three possible regions for key points: flat area, edge or corner.]{Three possible regions that can be used as keypoints. R1 cover a complete flat region. R2 is located on an edge. R3 is placed on a corner, this is the location that can be better recognised.}
		\label{fig:kpMatch_regions}
	\end{minipage}
\end{figure}

\subsection{Key points matching}
The key points, or \textbf{feature points}, matching is a technique that compares two images and tries to recognise which are the common elements of the two. It is widely used to find small patterns in complex images, understand the variation of orientation, stitch panorama views and judge if the subject of two pictures could be the same.\\
This technique is based on key points. These elements are positions in the image that can be easily located and identified as unique. In~\Cref{fig:kpMatch_regions} are shown three regions, a flat area, an edge and a corner. The most recognizable point is the one placed at the corner. If an object appears in two images a key points should be recognised in both the appearances.

\subsubsection*{Technique definition and known algorithms}
This technique is divided into two modules:
\begin{itemize}
	\item \textbf{Feature localization}: this module, working on a single image at a time, is internally divided into:
	\begin{itemize}
		\item \textbf{Feature detection}: given the raw image the goal is to identify all the corners that can be easily re-identified in another image.
		\item \textbf{Feature description}: a small area containing few pixels cannot be easily matched. So, the descriptor needs to take each identified key points and normalize them in order to be independent from the image conditions like illumination and orientation. These key points enriched with additional information are the feature points.
	\end{itemize}
	A visualization of the feature localization is shown in~\Cref{fig:kpMatch_tesla}.\\
	To solve this task several methods were used\footnote{Note that since the mechanics of the methods are similar and these techniques has not actually been implemented in the thesis the algorithms reported here are not explained.}:
	\begin{itemize}
		\item \underline{SIFT (Scale-Invariant Feature Transform)}\cite{sift} is able to manage both rotations and scale variation of the feature points.
		\item \underline{SURF (Speeded-Up Robust Features)}\cite{surf} is the advanced version of SIFT with all its pros and, in addition, has a lower processing time and produce an extremely high number of feature points.
		\item \underline{ORB (Oriented FAST and Rotated BRIEF)}\cite{orb} is the only algorithm of the three that is not patented. It is a combination of two sub methods:
		\begin{itemize}[\ding{228}] %method to insert custom symbol in itemize (\ding(228) = the full >)
			\item \underline{FAST (Features from Accelerated Segment Test)}\cite{fast} is a method completely focused on pure speed.
			\item \underline{BRIEF (Binary Robust Independent Elementary Features)}\cite{brief} is an algorithm extremely sensitive to noise, it uses Gaussian kernels to remove it.
		\end{itemize}
	\end{itemize}

	\item \textbf{Feature matching}: the third module of this technique works on two images after the feature points were computed. The goal is to match the features of the two images together. A match can only be one by one. A feature point should have exactly one corresponding point in the other image and vice versa, to be a correct and reliable connection. If a point is linked to more than one point on the other image, it means that the connection works with multiple inaccurate key points hardly recognisable.\\
	In~\Cref{fig:kpMatch_notreDame} is presented an example working on two similar images. All the correct matches, drawn as \textbf{\textcolor{green}{green}} lines, can be used to assign a probability that represents how likely the two images contain the same "object". Moreover the matches can also be used to understand which type of 3D transformation can be applied to an image to convert it into the other one.\\
	\\
	The existing algorithms for the matching are:
	\begin{itemize}
		\item \underline{BFM (Brute Force Matcher)}. This is the naive solution. All the possible combinations of points of the two different images are being tried. Obviously it is a precise method because it does not work under assumptions but if the number of key points is high this method is infeasible.
		\item \underline{FLANN (Fast Library for Approximate Nearest Neighbors) matcher}\cite{flann}. Differently from BFM, this method is optimized to work even with a large number of features. The optimization is based on the local search of K-Nearest Neighbors.
	\end{itemize}
\end{itemize}
\begin{figure}[!h]
	\centering
	\begin{subfigure}[!h]{0.7\textwidth}
		\includegraphics[width=\linewidth]{images/recognition/kpMatch_tesla}
		\caption{The feature localization module applied to a Tesla. The key points are spread along the edges of the car and the background. Instead, the road has no points, because the pattern is often repeated, hence it is not reliable.}
		\label{fig:kpMatch_tesla}
	\end{subfigure}
	\\
	\begin{subfigure}[!h]{0.7\textwidth}
		\includegraphics[width=\linewidth]{images/recognition/kpMatch_notreDame}
		\caption{The feature matcher is used to align two images of Notre Dame de Paris. Since the images are very clear a lot of points are paired correctly (\textbf{\textcolor{green}{green}} lines), while a few of them are not (\textbf{\textcolor{red}{red}} lines).}
		\label{fig:kpMatch_notreDame}
	\end{subfigure}
	\captionsetup{margin=0.5cm}
	\caption[Two examples of the key points matching modules.]{Examples of the two modules of the key points matching: feature localization and matcher.}
	\label{fig:kpMatch}
\end{figure}


\subsubsection*{Key points matching applied to people}
The initial idea was to apply the key points matching to people. The tests were done on the \textbf{PRID450 (Person Re-IDentification)}\cite{prid450} dataset that contains thousands of images of cropped people walking outdoors. The dataset is constructed with multiple shots of the same person in different moments and prospectives.\\
The idea has two big problems:
\begin{itemize}
	\item The images cropped around the people have a very low resolution. The result is that the details that could distinguish a person from another one cannot be visible, or better cannot be recognised.
	\item Humans present a high deformable-body, with a surface (clothes) that continuously change aspect. Instead, the key points matching is designed for a pattern that is repeated often and clearly. The consequence of this, is a matching that works as if it was random.
\end{itemize}
In~\Cref{fig:kpMatch_samples} are shown some examples that visually demonstrate the unreliability of this technique applied to humans.

\begin{figure}[!h]
	\centering
	\includegraphics[width=\linewidth]{images/recognition/kpSample_all}
	\captionsetup{margin=0.5cm}
	\caption[Key points matching samples of people comparison.]{Some matching samples show that key points matching easily fails under the conditions previously presented. There are multiple wrong aspects: people who present no key points, objects such as bags that have plenty of features, matches that connect completely different parts of the body like shoulders with legs. Even with the same subject with almost the same position (bottom-left couple of images), the algorithm fails with most of the points. The exception is the bottom-right image that has a perfect matching, but the two pictures are exactly the same one. Therefore it is not a reliable test.}
	\label{fig:kpMatch_samples}
\end{figure}


\section{KNN (K-Nearest Neighbors) with images into N-dimensional space}\label{sec:knn}
Since that offline algorithms cannot be used, the official solution is based on online algorithms. The chosen method is KNN (K-Nearest Neighbors) classifier. It is a widely used method that is based on the proximity of points into space. Where each point has a class. When a new point is classified, the K nearest known points are found and the class for the new incoming is chosen according to the majority of classes of the K selected points (an example is shown in~\Cref{fig:howItWorks_knn}).\\
About this algorithm, some aspects should be considered:

\begin{itemize}
	\item KNN is a native online method. The difference between online and offline is about computing the training in advance or not, but KNN has no training phase. The training only requires to store the known points with the associated labels and this is done with almost no cost.
	\item The elaboration works with points and not images, this remarkably reduces the computational classification cost.\\
	On the other hand, an image should be collapsed to a point, meaning that a good representation should be used to not lose important information.
	\item The classification steps should compare the new cropped frame, transformed into a point, with all the other known points. So, a very high number of stored points will slow down the execution. In our real-time scenario, this will occur only if the tracking will last for an extremely long period, but this will not happen.
\end{itemize}
To apply KNN, the challenging task of converting images into representative points should be solved.
\begin{figure}
	\centering
	\begin{minipage}{.44\textwidth}
		\centering
		\includegraphics[width=1\linewidth]{images/recognition/howItWorks_knn}
		\captionof{figure}[Example of application of the KNN classifier.]{Example of the mechanics of KNN. The new point (question mark) will be classified as $B$ if the 3 nearest neighbours are considered. It will be classified as $A$ if K=7 is used.}
		\label{fig:howItWorks_knn}
	\end{minipage}
	\begin{minipage}{.54\textwidth}
		\centering
		\includegraphics[width=1\linewidth]{images/recognition/howItWorks_resNet}
		\captionsetup{margin=0.5cm}
		\captionof{figure}[The residual block of ResNet.]{The residual block of ResNet. The identity connection allow to the output of the precedent level, to skip two convolution layers. The result $f(x)+x$ is the combination of the convolutions $f(x)$ combined with the identity $x$.}
		\label{fig:howItWorks_resNet}
	\end{minipage}
\end{figure}


\subsection{Image classifiers for representative points from images} \label{sec:classifiers}
We have chosen to use CNN to create the representative points that should describe an entire image. Unluckily, it does not exist an explicit field of study that aims to create this kind of points. The solution relies into the adaptation of a widely explored machine learning challenge called \textbf{image classification} (\Cref{fig:imgAnalysisType}). The goal is to predict which kind of elements exist inside the picture.\\
The image classifiers based on CNN generally work as follows:
\begin{enumerate}
	\item The input picture is resized to standard dimensions. In addition, colours and lights are normalized.
	\item The image is elaborated with multiple blocks of convolution layers.
	\item All the features extracted with the convolutions are collapsed, with a "\textit{flatten}" operation, into an array of thousands of elements.
	\item In the end, this vector is eventually reduced and the final predictions, one for each output class, are generated.
\end{enumerate}
The goal is to generate a point from the input image of CNN. The fourth part collapses all the elaborated information into predictions that vary according to the context of the application. Instead, the third level produces an array: a list of N numbers that can be seen as coordinates of a point into an \textbf{N-dimensional space}. KNN works independently from the space dimensions, so, it does not matter how many features are produced by the CNN based algorithms.\\
\\
The classifiers chosen are the DNNs (Deep Neural Networks) \textbf{GoogLeNet} and \textbf{ResNet}. A DNN has the capability to produce better results than a NN. On the other hand, the huge number of parameters used should be tuned during the training phase. This calibration of the values is extremely hard on small datasets due to the \textbf{vanishing gradient problem}. Therefore, both classifiers introduce a novelty aiming at solving the problem connected to the depth of the network.

\subsubsection*{ResNet (Residual Network)}
ResNet\cite{resNet_paper} is built on the idea of "skip blocks of layers". The residual block is shown in~\Cref{fig:howItWorks_resNet}.\\
A "\textit{plain CNN}" has convolutions stacked one after the other, in this case, there is an additional element: the \textbf{identity connection}. It means that no filters are applied and the input of this block is shifted two layers down. This connection is used to propagate the information deep into the CNN without modifying them. The advantage is that the input image is preserved through the network and it is not affected by an elaboration that lasts for several layers (more than 100). This novelty is very important for small training sets that are not able to fine-tune all millions of parameters of the DNN.\\
The authors of the original paper has created multiple models characterized by different depths, for this thesis we have chosen ResNet50\cite{resNet_model}. This model produces a representation point in \textbf{2048 dimensions}.

\subsubsection*{GoogLeNet (Google Le-Network)}
GoogLeNet\cite{googLeNet_paper} is based on a new convolution scheme called \textbf{Inception module}. The name and the goal of the module come from the quote “\textit{We need to go deeper}”\footnote{Quote of the film Inception (\href{https://i.kym-cdn.com/photos/images/newsfeed/000/531/557/a88.jpg}{meme}).}.\\
The naive version of the inception module parallelizes three different convolution filters (1x1, 3x3, 5x5) and a \textit{max-pooling filter}. This special elaboration reduces the depth of the CNN while preserving its potentiality. On the other hand, the number of parameters is still huge. The official version of the inception module (\Cref{fig:howItWorks_inceptionModule}) stacks each filter (3x3, 5x5 and max-pooling) with a 1x1 convolution to reduce, by a factor of 10, the overall number of parameters.\\
We have used the model\cite{googleNet_model} proposed together with the paper. This GoogLeNet implementation produces a representation point in \textbf{1024 dimensions} (half of ResNet50).
\begin{figure}[!h]
	\centering
	\includegraphics[width=0.8\linewidth]{images/recognition/howItWorks_inceptionModule}
	\captionsetup{margin=0.5cm}
	\caption{The inception module presented with the GoogLeNet model.}
	\label{fig:howItWorks_inceptionModule}
\end{figure}

\subsubsection*{SSP-ReID (Saliency-Semantic Parsing Re-IDentification)}
SSP-ReID\cite{ssp_reID} is a useful technique specifically designed to improve the potentialities of the image classifiers when applied to bounding boxes of people. This method is ideated to create a representative array of features.\\
The model is based on a "\textit{CNN backbone}" that can be selected from a wide set of pre-existing CNNs such as ResNet or GoogLeNet. The improvement comes from additional pieces of information that are processed together with the backbone.\\
The extra information generated (shown in~\Cref{fig:howItWorks_sspReID}) are:
\begin{itemize}
	\item \textbf{Saliency} (\Cref{fig:sub_saliency}): this technology aims at isolating all the pixels that appear at "first glance" to the human eyes. The set of these pixels are the one that may highlight some key aspects of a person such as a bag, the clothes or other.
	\item \textbf{Semantic Parsing} (\Cref{fig:sub_semanticParsing}): the body of the person is divided into 5 sections and these are elaborated separately. Each body part can have its own characteristics.\\
	The 5 sections are: head, upper body, lower body, shoes and complete body.
\end{itemize}
At the moment this method is not integrated into the thesis project but it can be a great choice for future improvements.

\begin{figure}[!h]
	\centering
	\begin{subfigure}{0.33\textwidth}
		\includegraphics[width=\linewidth]{images/recognition/ssp_saliency}
		\caption{The saliency elaboration, applied to the woman, highlights the presence of a bag.}
		\label{fig:sub_saliency}
	\end{subfigure}
	\begin{subfigure}{0.66\textwidth}
		\includegraphics[width=\linewidth]{images/recognition/ssp_semanticParsing}
		\captionsetup{margin=0.5cm}
		\caption{The semantic parsing elaboration divides the body of the child in the 5 sections: head, upper body, lower body, shoes and complete body.}
		\label{fig:sub_semanticParsing}
	\end{subfigure}
	\captionsetup{margin=0.5cm}
	\caption{Examples of saliency and semantic parsing elaborations.}
	\label{fig:howItWorks_sspReID}
\end{figure}

\subsection{Examples of KNN applied to people recognition}
The intuition of using KNN has been tested on the \textbf{Market1501}\cite{market1501} dataset. Similarly to the PRID450 dataset, also this one contains sets of images captured from different perspectives and in different moments of hundreds of people. Each real person has it is own ID so different images of the same subject are associated to the same ID.\\
Some of the results of the experiments are shown in~\Cref{fig:knn_reID_examples}. This elaboration is done by selecting two small datasets, one made of $18$ images of $2$ people and the other one made of $99$ images of $11$ people. For each person were used $9$ pictures. KNN was "trained"\footnote{The training of KNN consists of storing data and nothing more.} with the representative points extracted from the images of the datasets. Then, the queries (images of people) were used to retrieve the most similar people. This was done by generating the representative points of the queries and for each one, the K closest points are retrieved together with the associated images.\\
\\
It is important to focus on the ratio between correct and wrong responses, \textbf{\textcolor{green}{green}} and \textbf{\textcolor{red}{red}} respectively. In the test elaborated with ResNet50 (\Cref{fig:knn_resNet_s9n99}) there are almost 50\% and 50\% of wrong and correct responses. In the test elaborated with GoogLeNet (\Cref{fig:knn_googleNet_s9n18}) there is only one false prediction over $14$. Despite the different algorithms used the results are independent of them.\\
In fact, the key difference is that in~\Cref{fig:knn_resNet_s9n99} there are 11 classes, so $9*1=9$ samples of the correct person and $9*10=90$ samples of the wrong one. Whereas in~\Cref{fig:knn_googleNet_s9n18} there are only 2 classes so $9$ samples against $9$. This different ratio between positive and negative training examples affects the result of the predictions.\\
In this thesis, we are dealing only with $2$ classes: the leader and the other people. Hence, we are interested in the scenario with $18$ images that works extremely well.\\
\\
Lastly, for the integration of people recognition, with the tracking and detection modules, we only need to know if the query belongs to a class or to another one. This choice is based on the most likely class on the first K\footnote{In case of 2 classes, an odd number is chosen to be the value of K.}  nearest neighbours of the query, so if the majority is \textbf{\textcolor{green}{green}} or \textbf{\textcolor{red}{red}}.

\begin{figure}[!h]
	\centering
	\begin{subfigure}{1\textwidth}
		\includegraphics[width=\linewidth]{images/recognition/knn_resNet_s9n99}
		\captionsetup{margin=0.5cm}
		\caption{KNN applied to images elaborated with ResNet50. The training was done with 11 real people and 9 images of each one, for a total of 99 pictures. Here are shown 3 queries with the 8 most similar people.}
		\label{fig:knn_resNet_s9n99}
	\end{subfigure}
	\par\bigskip %vertical spacing
	\begin{subfigure}{1\textwidth}
		\includegraphics[width=\linewidth]{images/recognition/knn_googleNet_s9n18}
		\captionsetup{margin=0.5cm}
		\caption{KNN applied to images elaborated with GoogLeNet. The training was done with 2 real people and 9 images of each one, for a total of 18 pictures. Here are shown 2 queries with the 7 most similar people.}
		\label{fig:knn_googleNet_s9n18}
	\end{subfigure}
	\captionsetup{margin=0.5cm}
	\caption[KNN applied with image classifier to solve the person re-identification task.]{In this picture are shown queries computed on the KNN classifier that has pre-processed small datasets of images of people. The query (top-left bounding box with \textbf{\textcolor{blue}{blue}} contour) is used to extract from the database the most similar 7/8 pre-analysed people. The \textbf{\textcolor{green}{green}} contour means that the extracted person is correct, whereas if it is wrong the \textbf{\textcolor{red}{red}} is used.}
	\label{fig:knn_reID_examples}
\end{figure}	\clearpage
		\chapter{Solution} \label{cha:solution}
This chapter is focused on the integration of the three modules of this thesis: detection, recognition and tracking. The explanation is based on the flow of the code, with special attention at the choices done and implementation details. The source code of the entire project is available on github\cite{projectSourceCode}.

\section{Wrapper function: follow} \label{sec:follow}
The code is entirely managed with a single class called \textbf{Follower}. This structure only requires to be initialized and then calling a "\textit{follow}" method (\Cref{alg:follow}) every time a new frame needs to be processed.\\
The class internally loads a new frame \li{2} from the webcam or, optionally, from a stored video. Both the sources can be used in "real-time". In fact, if the video is used some frames are internally discarded to simulate the losing of images due to a slow processing rate. Therefore the class \textit{Follower} can be analysed with both real-time tests but also recorded experiments. This feature is useful to replicate scenarios where the code have failed.\\
\\
The wrapper method \textit{follow} has only one task. It measures the elapsed time from the beginning of the tracking \li{4} and, according to this parameter, the \textit{slow start} phase (\Cref{sec:slowStartPhase}) is executed \li{5}  if less than X seconds are gone. Otherwise, the \textit{track leader} phase (\Cref{sec:trackleaderPhase}) is called \li{7}.

\begin{lstlisting}[captionpos=b, 
	caption={It is the pseudocode of the wrapper function that should grab the new incoming frames and redirect them to the first or second phase according to the time elapsed from the tracking begin.}, 
	label=alg:follow
	]
follow() -> position:
	frame = grab_newFrame()
	
	if elapsedTime() > phase1_length: #phase 1
	%*$\lfloor$*)	position = slowStartPhase(frame)
	else:							  #phase 2
	%*$\lfloor$*)	position = followPhase(frame)

	return position
\end{lstlisting}


\section{First phase: slow start} \label{sec:slowStartPhase}
This function is a novelty that we have chosen to introduce to empower the performances of the overall algorithm. The pseudo code of this method is provided at~\Cref{alg:slowStartPhase} while an example is shown in~\Cref{fig:slowStart}.\\
This project, differently from the traditional trackers and similarly to TLD (\Cref{sec:tld}), is based on an online learning classifier. Hence this phase is designed to immediately train KNN a little bit. This first generated knowledge will be then increased in the \textit{track leader} phase. KNN can classify the representative points that come from the bounding boxes of people in new frames, only according to other representative points previously archived. The \textit{slow start} phase is used to collect all the bounding boxes used to create the set of positive representative points into the N-dimensional space of KNN.\\
This elaboration works based on the assumption that during this first phase "\ul{The leader is the most important visible person}". The \textit{slow start} computes the detection of the visible people \li{3} and if one or more exists\li{5}, the leader is chosen as the bounding box, with the biggest area\li{6}. The X and Y pixel coordinates are computed to be retrieved \li{7} and the leader's box is given to KNN \li{8}. Simultaneously, a negative sample is randomly picked up from a database and it is also given to KNN \li{9}. This double fed is done to train KNN with a balanced number of positive and negative samples, in order to exploit the potentialities shown in~\Cref{fig:knn_googleNet_s9n18}. The negative samples come from a custom version of the Market1501 dataset\cite{market1501}, called \textbf{NegativePeople}, that we have created ad hoc for this purpose.\\
\\
The \textit{slow start} phase is executed for a period that can last for 3 up to 20 seconds or more. The value of this hyper-parameter influence the size of samples known by KNN when the official track starts. The longer this phase the better. The default value is set to 5 seconds. Alternatively, another possibility is that the first phase can be interrupted after that X points are given to KNN, but this does not guarantee a precise slot of time so this variant was discarded.

\begin{lstlisting}[captionpos=b, 
	caption={It is the pseudocode of the first phase. The function \textit{slow start} computes only detections in order to train the KNN people classifier.}, 
	label=alg:slowStartPhase
	]
slowStartPhase(frame) -> position:
	position = default
	boxes = detector.detectPeople(frame)
	
	if len(boxes)>=1:
	%*$\mid$*)	box = biggestAreaBB(boxes)
	%*$\mid$*)	position = getPosition(box)
	%*$\mid$*)	knn.addPositive(box)
	%*$\lfloor$*)	knn.addNegative(pickOneNegative())
	
	return position
\end{lstlisting}
\begin{figure}
	\centering
	\begin{minipage}{.49\textwidth}
		\centering
		\includegraphics[width=1\linewidth]{images/solution/slowStart}
		\captionsetup{margin=0.2cm}
		\captionof{figure}[Frame of the slow start phase.]{Frame of the \textit{slow start} phase where the detection only is working. The leader has a \textbf{\textcolor{blue}{blue}} rectangle, and eventually other people are in \textbf{\textcolor{gray}{gray}}.}
		\label{fig:slowStart}
	\end{minipage}
	\begin{minipage}{.49\textwidth}
		\centering
		\includegraphics[width=1\linewidth]{images/solution/leaderSubjectOk}
		\captionsetup{margin=0.2cm}
		\captionof{figure}[A perfect detection example.]{A perfect detection where the two people are correctly recognised as leader (positive=\textbf{\textcolor{green}{green}}) and random person (negative=\textbf{\textcolor{red}{red}}).}
		\label{fig:doubleDetectionOk}
	\end{minipage}
\end{figure}


\section{Second phase: track leader} \label{sec:trackleaderPhase}
This function is the core of the entire project. The overall scheme of the pseudocode is shown at~\Cref{alg:trackleaderPhase}. In addition, in~\Cref{fig:sequenceTracking} there is a sequence of frames taken from a video clip where the robot is following the leader while it is shortly occluded twice by another person.\\
This phase works as follows:\\
\begin{itemize}
	\item The detection is performed \li{7} only one time every X frames \li{5}(details in~\Cref{sec:ratioDetectTrack}) and if the position of the leader is unknown \li{6}.\\
	The position is unknown immediately after the \textit{slow start} phase \li{2}, after the end of the tracking \li{29} and after a failed detection \li{17}.\\
	A sample detection over two people is shown in~\Cref{fig:doubleDetectionOk}.
	
	\item All the detections are elaborated \li{10} to check if are close to the last known position \li{11} and so can be kept or not (the drift optimization details are in~\Cref{sec:driftOptimization}). In addition, the KNN classifier is used \li{12} to accurately understood which detections contain the leader and which not (errors tolerance explained in~\Cref{sec:knnToleranceToFN}).\\
	Then, the false predictions are added to KNN \li{15}, while the positive ones are stored for more controls \li{13}.
	
	\item All the boxes that seem to contain the leader are further analysed \li{17-18}. The official prediction is chosen as the closest feasible box to the last known position \li{19} (more details in~\Cref{sec:multipleleaders}). Based on this final choice the tracked is re-initialized \li{20}.
	
	\item The new position is computed \li{22} and the elaborated bounding boxes are stored into KNN according to their content \li{23-24}. If only one detection was initially found \li{25}, and it was the leader, KNN is fed with a negative sample \li{26} coming from the NegativePeople dataset.
	
	\item After the initialization, the tracking is updated with new frames, one at a time \li{30}. The retrieved bounding box once converted into a position \li{31} is returned to the \textit{follow} function, both for detection and tracking \li{33}.
\end{itemize}
Special conditions and key aspects to focus on, follow in the next sections.

\begin{lstlisting}[captionpos=b, 
	caption={It is the pseudocode of the second phase. The function \textit{track leader} alternatively runs the detection and tracking modules to constantly know the position of the leader.}, 
	label=alg:trackleaderPhase
	]
trackleaderPhase(frame) -> position:
	static stopDetections = False
	position = default
	
	if onceEveryKTimes(10) 
	%*$\mid$*)  and not stopDetections: 					 #detection
	%*$\mid$*)	boxes = detector.detectPeople(frame)
	%*$\mid$*)	
	%*$\mid$*)	boxesOfleader = []
	%*$\mid$*)	foreach box in boxes:
	%*$\mid$*)	%*$\mid$*)	if checkDriftProximity(box)	  #drift optimization
	%*$\mid$*)	%*$\mid$*)	%*$\mid$*)   and knn.classify(box)==positive: #recognition
	%*$\mid$*)	%*$\mid$*)	%*$\lfloor$*)	 boxesOfleader.add(box)
	%*$\mid$*)	%*$\mid$*)	else:
	%*$\mid$*)	%*$\lfloor$*)	%*$\lfloor$*)	knn.addNegative(box)
	%*$\mid$*)			
	%*$\mid$*)	stopDetections = (len(boxesOfleader) > 0)
	%*$\mid$*)	if stopDetections:
	%*$\mid$*)	%*$\mid$*)	boxOfleader = pickClosestPosition(boxesOfleader) 
	%*$\mid$*)	%*$\mid$*)	tracker.initialize(frame, boxOfleader)
	%*$\mid$*)	%*$\mid$*)	
	%*$\mid$*)	%*$\mid$*)	position = getPosition(boxOfleader)
	%*$\mid$*)	%*$\mid$*)	knn.addPositive(boxOfleader)			 
	%*$\mid$*)	%*$\mid$*)	knn.addNegative(boxesOfleader except boxOfleader)
	%*$\mid$*)	%*$\mid$*)	if len(boxes) == 1:
	%*$\mid$*)	%*$\lfloor$*)	%*$\lfloor$*)	knn.addNegative(pickOneNegative())
	%*$\lfloor$*)		
	else: 										 #tracking
	%*$\mid$*)	stopDetections = False
	%*$\mid$*)	box = tracker.updateRegion(frame)
	%*$\lfloor$*)	position = getPosition(box)
	
	return position
\end{lstlisting}
\begin{figure}[!h]
	\centering
	\includegraphics[width=\linewidth]{images/solution/sequenceTrackOk}
	\captionsetup{margin=0.5cm}
	\caption[A video clip frames sequence of the tracking.]{A sequence of images representing a video clip while the robot is following the leader. The coloured rectangles in the images have different meanings. \textbf{\textcolor{green}{Green}} is the detection recognised as the leader. \textbf{\textcolor{red}{Red}} is a detection recognised as a not important person. \textbf{Black} is the tracker while following the leader.\\
		In this sample the leader is hidden twice and both times the algorithm is able to detect it back again and continue the tracking.}
	\label{fig:sequenceTracking}
\end{figure}


\subsection{Detection and tracking ratio} \label{sec:ratioDetectTrack}
It is fundamental to precisely define the alternation of the detection and the tracking along with the execution of the general method. This calibration is a trade-off among processing speed and localization accuracy. As shown in~\Cref{tab:detectionPerformances} and in~\Cref{tab:trackersFPS} the processing speed of the two modules is completely different. The detection is much slower compared to the tracking. Therefore, the highest FPS rate is reached with a tracker only solution, on the other hand, the highest localization accuracy is achieved with detection only technique. Note that with only detection and multiple subjects, the leader is identified thanks to the recognition module.\\
The disclaimer is to choose the minimum required FPS rate and then hope that the accuracy is enough. In our case, we have fixed $5$ FPS as a target. To respect this limit the detection is executed once every $10$ frames. It is important to remember that if a single detection fails or the leader is not found, the tracker cannot be started. Consequently, from that moment on the frames are processed with detection-only. This recover procedure run at low FPS but the leader is momentarily lost, hence, it is not important.\\
\\
If the high FPS needs to be reached the solution may consist of changing this mechanic. The detector might be started only after the tracker reports that it lost the leader. This will reduce the number of detections and improve the overall FPS rate. Unfortunately, a lot of trackers are not precisely able to recognize when the tracked subject has been lost, hence the implementation is not straightforward, and at the moment has not been written yet.


\subsection{Drift tolerance optimization} \label{sec:driftOptimization}
The idea of this optimization comes from two conditions that should be managed. On one hand, the drift problem that is one of the main weaknesses of the trackers, in fact over a long video sequence it can be a huge problem. On the other hand, the proximity assumption (\Cref{fig:challenge_proximity}) that allows the tracked subject to move only for a limited number of pixels per frame.\\
These two conditions combined can be used to wisely classify the new detections once the tracker has been stopped. If a new detection is too far away compared to the last known position of the leader, it cannot be the leader itself. Otherwise, the recognition module should be used to normally predict the class of the new bounding box.\\
We have defined the tolerance of the movement as:
$$d = t\cdot s\cdot\left(\frac{w}{100}\right)^{2}$$
In the above formula the symbols represent:
\begin{itemize}
	\item \textbf{d} is the distance allowed.
	\item \textbf{t} is the time elapsed from the last correct detection of the leader. It is used to manage the drift problem independently of the ratio between tracking and detection.
	\item \textbf{s} is an empirical scale factor, experimentally measured to be around $0.05$. It is the hyper-parameter to manage this optimization. 
	\item \textbf{w} is the width of the last bounding box during the tracking. It is used to simulate the distance of the leader from the camera. Compared to a far detection, a close leader has a bigger bounding box hence the multiplying factor is greater. This difference is fundamental to manage the fast 2D movement of a close subject.
\end{itemize}
In~\Cref{fig:driftOptimizationFail} is shown an example where a person is immediately classified as negative because it is too far away. Instead, in~\Cref{fig:driftOptimizationOk} the person is inside the tolerance and further analysis with KNN has classified this person as the leader.

\begin{figure}[!h]
	\centering
	\begin{subfigure}[!h]{0.49\textwidth}
		\includegraphics[width=\linewidth]{images/solution/driftOptimizationFail}
		\captionsetup{margin=0.5cm}
		\caption{The person is classified as Negative because it is outside of the \textit{drift tolerance} circle.}
		\label{fig:driftOptimizationFail}
	\end{subfigure}
	\begin{subfigure}[!h]{0.39\textwidth}
		\includegraphics[width=\linewidth]{images/solution/driftOptimizationOk}
		\caption{The person is classified as Positive because it is outside of the \textit{drift tolerance} circle.}
		\label{fig:driftOptimizationOk}
	\end{subfigure}
	\captionsetup{margin=1.4cm}
	\caption[Two samples of the \textit{drift tolerance} optimization.]{Two samples of the \textit{drift tolerance} optimization represented as a \textbf{\textcolor{orange}{yellow}} circle, centred on the last known position (a small \textbf{black} circle).}
	\label{fig:driftOptimization}
\end{figure}

\subsection{Multiple leaders corner case} \label{sec:multipleleaders}
It may happen that there are two people both inside the \textit{drift tolerance} circle and both are classified as the leader from KNN. This of course is real scenario impossible (a person cannot be duplicated) hence only one can be chosen as the right one. Further analysis can be used but we have chosen a simpler idea: \textit{"The closest subject to the last know position will be the official leader"}.\\
In~\Cref{fig:leaderSubjectDoubleMatch} there is a sample scenario registered before the introduction of this elaboration.
\begin{figure}[!h]
	\centering
	\begin{subfigure}[!h]{0.49\textwidth}
		\includegraphics[width=1\linewidth]{images/solution/leaderSubjectDoubleMatch}
		\captionsetup{margin=0.2cm}
		\caption{Two recognised leader, instead one detection is a random person.}
		\label{fig:leaderSubjectDoubleMatch}
	\end{subfigure}
	\begin{subfigure}[!h]{0.49\textwidth}
		\includegraphics[width=1\linewidth]{images/solution/leaderSubjectDoubleFail}
		\captionsetup{margin=0.2cm}
		\caption{Two recognised random people, instead one detection is the leader.}
		\label{fig:leaderSubjectDoubleFail}
	\end{subfigure}
	\captionsetup{margin=0.5cm}
	\caption{Two failed scenarios where the two people are wrongly recognised.}
	\label{fig:qqq}
\end{figure}

\subsection{KNN tolerance against false-negative} \label{sec:knnToleranceToFN}
The previous section introduces the failure of the KNN classifier that predicts a leader in exceed. This is a false-positive classification. As explained this is miss prediction can be managed. On the other hand, a false-negative classification is a much bigger problem. It happens when the leader is classified as negative, an example is shown in~\Cref{fig:leaderSubjectDoubleFail}.\\
This wrong prediction is complex because, after the classification, the generated representative point and its new label are fed into KNN. This false-negative represent a point wrongly classified. However, also a false-positive is fed into KNN but if there are multiple leaders further analysis can reduce them. Instead, a leader classified as negative cannot be re-evaluated hence it cannot be converted into a correct prediction, it will be an error forever.\\
Due to the mechanics of KNN, a small set of wrong classified points can cause a lot of wrong classifications in future analysis. Therefore we absolutely want to avoid false-negative predictions.		\clearpage
		\chapter{Conclusions} \label{cha:conclusions}
In this thesis project, a wide variety of algorithms have been explored, implemented and tested. The general flow of the program has been changed often ending up in the version that is presented in the previous pages.\\
\\
The result is a software application that satisfies all the requirements that were chosen in the beginning. The algorithm is able to be executed in real-time at $5$ FPS on not top quality hardware, both on an Intel Core i5 CPU and on an Nvidia Jetson TX2 GPU. The fast execution allows the robot to physically follow the leader in the real environment, while the software is tracking it precisely. Moreover, the tracking can last for several minutes without any problem\footnote{We have tested the algorithm only on the minutes time scale, but there are no substantial reasons that prevent the software to be executed even for hours. It is only a matter of tests that were not done. Hence, we cannot officially state that this algorithm is able to run for hours.}.\\
The main known weakness is against the false-negative predictions of the people classifier, as widely explained. Hence, according to the \textit{confusion matrix measures}, we can state that our combination of methods, to work precisely, requires a high \textit{recall} value for the people classifier module.\\
\\
From another point of view, referring to the architecture of the overall project, we have chosen a structure that aims at being modular. During the study of the technologies applied to the project, we have realized that plenty of them can be used in order to solve the tracking challenge proposed in this dissertation. In addition, computer vision technologies are rapidly changing. In fact, the papers with the majority of the methods used in this work, were written in the last few years. Based on the data, these papers were published on average four years ago\footnote{We have measured this value by taking the average of the release dates of the papers of the algorithms that were identified as the best compromise along with all the dissertation. The average year of publication is 2016.}.\\
This means that in four years this project will probably be obsolete. To avoid this extremely rapid decay, it is fundamental to constantly replace the algorithms that are used as the core of the three main modules: detection, tracking and recognition. A modular structure allows to easily substitute a method without the necessity to change also the others. By doing this, we hope that this thesis project will be useful for more than a few years.


	\clearpage
      
    \endgroup
	
    % bibliografia in formato bibtex
    \phantomsection
    \addcontentsline{toc}{chapter}{Bibliography}	% aggiunta del capitolo nell'indice
    \bibliographystyle{unsrt}						% stile con ordinamento alfabetico in funzione degli autori
    \bibliography{biblio}
	%%%%%%%%%%%%%%%%%%%%%%%%%%%%%%%%%%%%%%%%%%%%%%%%%%%%%%%%%%%%%%%%%%%%%%
	%%%%%%%%%%%%%%%%%%%%%%%%%%%%%%%%%%%%%%%%%%%%%%%%%%%%%%%%%%%%%%%%%%%%%%
	%% Nota
	%%%%%%%%%%%%%%%%%%%%%%%%%%%%%%%%%%%%%%%%%%%%%%%%%%%%%%%%%%%%%%%%%%%%%%
	%% Nella bibliografia devono essere riportati tutte le fonti consultate 
	%% per lo svolgimento della tesi. La bibliografia deve essere redatta 
	%% in ordine alfabetico sul cognome del primo autore. 
	%% 
	%% La forma della citazione bibliografica va inserita secondo la fonte utilizzata:
	%% 
	%% LIBRI
	%% Cognome e iniziale del nome autore/autori, la data di edizione, titolo, casa editrice, eventuale numero dell’edizione. 
	%% 
	%% ARTICOLI DI RIVISTA
	%% Cognome e iniziale del nome autore/autori, titolo articolo, titolo rivista, volume, numero, numero di pagine.
	%% 
	%% ARTICOLI DI CONFERENZA
	%% Cognome e iniziale del nome autore/autori (anno), titolo articolo, titolo conferenza, luogo della conferenza (città e paese), date della conferenza, numero di pagine. 
	%% 
	%% SITOGRAFIA
	%% La sitografia contiene un elenco di indirizzi Web consultati e disposti in ordine alfabetico. 
	%% È necessario:
	%%   Copiare la URL (l’indirizzo web) specifica della pagina consultata
	%%   Se disponibile, indicare il cognome e nome dell’autore, il titolo ed eventuale sottotitolo del testo
	%%   Se disponibile, inserire la data di ultima consultazione della risorsa (gg/mm/aaaa).    
	%%%%%%%%%%%%%%%%%%%%%%%%%%%%%%%%%%%%%%%%%%%%%%%%%%%%%%%%%%%%%%%%%%%%%%
	%%%%%%%%%%%%%%%%%%%%%%%%%%%%%%%%%%%%%%%%%%%%%%%%%%%%%%%%%%%%%%%%%%%%%%
    
    % sezione Allegati - opzionale
	%\appendix
    %\input{allegati}

\end{document}

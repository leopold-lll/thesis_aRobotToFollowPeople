\chapter{Conclusions} \label{cha:conclusions}
In this thesis project, a wide variety of algorithms have been explored, implemented and tested. The general flow of the program has been changed often ending up in the version that is presented in the previous pages.\\
\\
The result is a software application that satisfies all the requirements that were chosen in the beginning. The algorithm is able to be executed in real-time at $5$ FPS on not top quality hardware, both on an Intel Core i5 CPU and on an Nvidia Jetson TX2 GPU. The fast execution allows the robot to physically follow the leader in the real environment, while the software is tracking it precisely. Moreover, the tracking can last for several minutes without any problem\footnote{We have tested the algorithm only on the minutes time scale, but there are no substantial reasons that prevent the software to be executed even for hours. It is only a matter of tests that were not done. Hence, we cannot officially state that this algorithm is able to run for hours.}.\\
The main known weakness is against the false-negative predictions of the people classifier, as widely explained. Hence, according to the \textit{confusion matrix measures}, we can state that our combination of methods, to work precisely, requires a high \textit{recall} value for the people classifier module.\\
\\
From another point of view, referring to the architecture of the overall project, we have chosen a structure that aims at being modular. During the study of the technologies applied to the project, we have realized that plenty of them can be used in order to solve the tracking challenge proposed in this dissertation. In addition, computer vision technologies are rapidly changing. In fact, the papers with the majority of the methods used in this work, were written in the last few years. Based on the data, these papers were published on average four years ago\footnote{We have measured this value by taking the average of the release dates of the papers of the algorithms that were identified as the best compromise along with all the dissertation. The average year of publication is 2016.}.\\
This means that in four years this project will probably be obsolete. To avoid this extremely rapid decay, it is fundamental to constantly replace the algorithms that are used as the core of the three main modules: detection, tracking and recognition. A modular structure allows to easily substitute a method without the necessity to change also the others. By doing this, we hope that this thesis project will be useful for more than a few years.



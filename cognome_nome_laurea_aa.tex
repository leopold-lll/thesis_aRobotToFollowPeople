%%%%%%%%%%%%%%%%%%%%%%%%%%%%%%%%%%%%%%%%%%%%%%%
%
% Template per Elaborato di Laurea
% DISI - Dipartimento di Ingegneria e Scienza dell’Informazione
%
% update 2015-09-10
%
% Per la generazione corretta del 
% pdflatex nome_file.tex
% bibtex nome_file.aux
% pdflatex nome_file.tex
% pdflatex nome_file.tex
%
%%%%%%%%%%%%%%%%%%%%%%%%%%%%%%%%%%%%%%%%%%%%%%%

% formato FRONTE RETRO
\documentclass[epsfig,a4paper,11pt,titlepage,twoside,openany]{book}
\usepackage{epsfig}
\usepackage{plain}
\usepackage{setspace}
\usepackage[paperheight=29.7cm,paperwidth=21cm,outer=1.5cm,inner=2.5cm,top=2cm,bottom=2cm]{geometry} % per definizione layout
\usepackage{titlesec} % per formato custom dei titoli dei capitoli

%%%%%%%%%%%%%%
% supporto lettere accentate
%
\usepackage[latin1]{inputenc} % per Windows;
%\usepackage[utf8x]{inputenc} % per Linux (richiede il pacchetto unicode);
%\usepackage[applemac]{inputenc} % per Mac.

\singlespacing

%\usepackage[italian]{babel}

\begin{document}

  % nessuna numerazione
  \pagenumbering{gobble} 
  \pagestyle{plain}

\thispagestyle{empty}

\begin{center}
  \begin{figure}[h!]
    \centerline{\psfig{file=logo_unitn_black_center.eps,width=0.6\textwidth}}
  \end{figure}

  \vspace{2 cm} 

  \LARGE{Dipartimento di Ingegneria e Scienza dell’Informazione\\}

  \vspace{1 cm} 
  \Large{Corso di Laurea in\\
    ...
    %Informatica
    %Ingegneria dell'Informazione e delle Comunicazioni
    %Ingegneria dell'Informazione e Organizzazione d'Impresa
    %Ingegneria Elettronica e delle Telecomunicazioni
  }

  \vspace{2 cm} 
  \Large\textsc{Elaborato finale\\} 
  \vspace{1 cm} 
  \Huge\textsc{Titolo\\}
  \Large{\it{Sottotitolo (alcune volte lungo - opzionale)}}


  \vspace{2 cm} 
  \begin{tabular*}{\textwidth}{ c @{\extracolsep{\fill}} c }
  \Large{Supervisore} & \Large{Laureando}\\
  \Large{......}& \Large{Stefano Leonardi}\\
  \end{tabular*}

  \vspace{2 cm} 

  \Large{Anno accademico .../...}
  
\end{center}



  \clearpage
 
%%%%%%%%%%%%%%%%%%%%%%%%%%%%%%%%%%%%%%%%%%%%%%%%%%%%%%%%%%%%%%%%%%%%%%%%%%
%%%%%%%%%%%%%%%%%%%%%%%%%%%%%%%%%%%%%%%%%%%%%%%%%%%%%%%%%%%%%%%%%%%%%%%%%%
%% Nota
%%%%%%%%%%%%%%%%%%%%%%%%%%%%%%%%%%%%%%%%%%%%%%%%%%%%%%%%%%%%%%%%%%%%%%%%%%
%% Sezione Ringraziamenti opzionale
%%%%%%%%%%%%%%%%%%%%%%%%%%%%%%%%%%%%%%%%%%%%%%%%%%%%%%%%%%%%%%%%%%%%%%%%%%
%%%%%%%%%%%%%%%%%%%%%%%%%%%%%%%%%%%%%%%%%%%%%%%%%%%%%%%%%%%%%%%%%%%%%%%%%%
  \thispagestyle{empty}

\begin{center}
  {\bf \Huge Ringraziamenti}
\end{center}

\vspace{4cm}


\emph{
  ...thanks to...
}

  \clearpage
  \pagestyle{plain} % nessuna intestazione e pie pagina con numero al centro

  
  % inizio numerazione pagine in numeri arabi
  \mainmatter

%%%%%%%%%%%%%%%%%%%%%%%%%%%%%%%%%%%%%%%%%%%%%%%%%%%%%%%%%%%%%%%%%%%%%%%%%%
%%%%%%%%%%%%%%%%%%%%%%%%%%%%%%%%%%%%%%%%%%%%%%%%%%%%%%%%%%%%%%%%%%%%%%%%%%
%% Nota
%%%%%%%%%%%%%%%%%%%%%%%%%%%%%%%%%%%%%%%%%%%%%%%%%%%%%%%%%%%%%%%%%%%%%%%%%%
%% Si ricorda che il numero massimo di facciate e' 30.
%% Nel conteggio delle facciate sono incluse 
%%   indice
%%   sommario
%%   capitoli
%% Dal conteggio delle facciate sono escluse
%%   frontespizio
%%   ringraziamenti
%%   allegati    
%%%%%%%%%%%%%%%%%%%%%%%%%%%%%%%%%%%%%%%%%%%%%%%%%%%%%%%%%%%%%%%%%%%%%%%%%%
%%%%%%%%%%%%%%%%%%%%%%%%%%%%%%%%%%%%%%%%%%%%%%%%%%%%%%%%%%%%%%%%%%%%%%%%%%

    % indice
    \tableofcontents
    \clearpage
    
    
          
    % gruppo per definizone di successione capitoli senza interruzione di pagina
    \begingroup
      % nessuna interruzione di pagina tra capitoli
      % ridefinizione dei comandi di clear page
      \renewcommand{\cleardoublepage}{} 
      \renewcommand{\clearpage}{} 
      % redefinizione del formato del titolo del capitolo
      % da formato
      %   Capitolo X
      %   Titolo capitolo
      % a formato
      %   X   Titolo capitolo
      
      \titleformat{\chapter}
        {\normalfont\Huge\bfseries}{\thechapter}{1em}{}
        
      \titlespacing*{\chapter}{0pt}{0.59in}{0.02in}
      \titlespacing*{\section}{0pt}{0.20in}{0.02in}
      \titlespacing*{\subsection}{0pt}{0.10in}{0.02in}
      
      % sommario
      \chapter*{Sommario} % senza numerazione
\label{sommario}

\addcontentsline{toc}{chapter}{Sommario} % da aggiungere comunque all'indice

Lorem ipsum dolor sit amet, consectetur adipiscing elit. Donec sed nunc orci. Aliquam nec nisl vitae sapien pulvinar dictum quis non urna. Suspendisse at dui a erat aliquam vestibulum. Quisque ultrices pellentesque pellentesque. Pellentesque egestas quam sed blandit tempus. Sed congue nec risus posuere euismod. Maecenas ut lacus id mauris sagittis egestas a eu dui. Class aptent taciti sociosqu ad litora torquent per conubia nostra, per inceptos himenaeos. Pellentesque at ultrices tellus. Ut eu purus eget sem iaculis ultricies sed non lorem. Curabitur gravida dui eget ex vestibulum venenatis. Phasellus gravida tellus velit, non eleifend justo lobortis eget.


  Sommario è un breve riassunto del lavoro svolto dove si descrive l'obiettivo, l'oggetto della tesi, le 
metodologie e le tecniche usate, i dati elaborati e la spiegazione delle conclusioni alle quali siete arrivati.  

Il sommario dell’elaborato consiste al massimo di 3 pagine e deve contenere le seguenti informazioni:
\begin{itemize}
  \item contesto e motivazioni 
  \item breve riassunto del problema affrontato
  \item tecniche utilizzate e/o sviluppate
  \item risultati raggiunti, sottolineando il contributo personale del laureando/a
\end{itemize}





%%%%%%%%%%%%%%%%%%%%%%%%%%%%%%%%%%%%%%%%%%%%%%%%%%%%%%%%%%%%%%%%%%%%%%%%%%
%%%%%%%%%%%%%%%%%%%%%%%%%%%%%%%%%%%%%%%%%%%%%%%%%%%%%%%%%%%%%%%%%%%%%%%%%%
%% Nota
%%%%%%%%%%%%%%%%%%%%%%%%%%%%%%%%%%%%%%%%%%%%%%%%%%%%%%%%%%%%%%%%%%%%%%%%%%
%% Sommario e' un breve riassunto del lavoro svolto dove si descrive 
%% l’obiettivo, l’oggetto della tesi, le metodologie e 
%% le tecniche usate, i dati elaborati e la spiegazione delle conclusioni 
%% alle quali siete arrivati.
%% Il sommario dell’elaborato consiste al massimo di 3 pagine e deve contenere le seguenti informazioni: 
%%   contesto e motivazioni
%%   breve riassunto del problema affrontato
%%   tecniche utilizzate e/o sviluppate
%%   risultati raggiunti, sottolineando il contributo personale del laureando/a
%%%%%%%%%%%%%%%%%%%%%%%%%%%%%%%%%%%%%%%%%%%%%%%%%%%%%%%%%%%%%%%%%%%%%%%%%%
%%%%%%%%%%%%%%%%%%%%%%%%%%%%%%%%%%%%%%%%%%%%%%%%%%%%%%%%%%%%%%%%%%%%%%%%%%      
      
      %%%%%%%%%%%%%%%%%%%%%%%%%%%%%%%%
      % lista dei capitoli
      %
      % \input oppure \include
      %
      \chapter{In ante nulla, vestibulum a}
\label{cha:intro}

Lorem ipsum dolor sit amet, consectetur adipiscing elit. Donec sed nunc orci. Aliquam nec nisl vitae sapien pulvinar dictum quis non urna. Suspendisse at dui a erat aliquam vestibulum. Quisque ultrices pellentesque pellentesque. Pellentesque egestas quam sed blandit tempus. Sed congue nec risus posuere euismod. Maecenas ut lacus id mauris sagittis egestas a eu dui. Class aptent taciti sociosqu ad litora torquent per conubia nostra, per inceptos himenaeos. Pellentesque at ultrices tellus. Ut eu purus eget sem iaculis ultricies sed non lorem. Curabitur gravida dui eget ex vestibulum venenatis. Phasellus gravida tellus velit, non eleifend justo lobortis eget. 
\cite{coulouris}

Donec eu ipsum id lorem consectetur luctus ac a nisi. Curabitur volutpat, metus id porta ultrices, felis lacus consectetur justo, ut gravida arcu ex in purus. Pellentesque vitae sapien ac nisl porttitor pellentesque eu sed elit. Sed maximus lectus eu eros ultricies accumsan. Quisque congue, nisi in dictum cursus, ante nisl molestie eros, in ultrices eros tellus sit amet augue. Interdum et malesuada fames ac ante ipsum primis in faucibus. Nam finibus leo sit amet purus vehicula, eget facilisis turpis convallis. Vivamus varius tincidunt turpis, id venenatis arcu maximus ut. Aenean euismod eros ac nibh facilisis, nec imperdiet ex suscipit.
\cite{dalal}


\section{Pellentesque habitant morbi tristique senectus}
\label{sec:context}

Lorem ipsum dolor sit amet, consectetur adipiscing elit. Donec sed nunc orci. Aliquam nec nisl vitae sapien pulvinar dictum quis non urna. Suspendisse at dui a erat aliquam vestibulum. Quisque ultrices pellentesque pellentesque. Pellentesque egestas quam sed blandit tempus. Sed congue nec risus posuere euismod. Maecenas ut lacus id mauris sagittis egestas a eu dui. Class aptent taciti sociosqu ad litora torquent per conubia nostra, per inceptos himenaeos. Pellentesque at ultrices tellus. Ut eu purus eget sem iaculis ultricies sed non lorem. Curabitur gravida dui eget ex vestibulum venenatis. Phasellus gravida tellus velit, non eleifend justo lobortis eget.
\cite{ictbusiness}
\cite{donoho}

\section{Nullam et justo vitae nisi}
\label{sec:problem}

Lorem ipsum dolor sit amet, consectetur adipiscing elit. Donec sed nunc orci. Aliquam nec nisl vitae sapien pulvinar dictum quis non urna. Suspendisse at dui a erat aliquam vestibulum. Quisque ultrices pellentesque pellentesque. Pellentesque egestas quam sed blandit tempus. Sed congue nec risus posuere euismod. Maecenas ut lacus id mauris sagittis egestas a eu dui. Class aptent taciti sociosqu ad litora torquent per conubia nostra, per inceptos himenaeos. Pellentesque at ultrices tellus. Ut eu purus eget sem iaculis ultricies sed non lorem. Curabitur gravida dui eget ex vestibulum venenatis. Phasellus gravida tellus velit, non eleifend justo lobortis eget.



      \chapter{Proin rhoncus a sapien in.}
\label{cha:789}
Lorem ipsum dolor sit amet, consectetur adipiscing elit. Donec sed nunc orci. Aliquam nec nisl vitae sapien pulvinar dictum quis non urna. Suspendisse at dui a erat aliquam vestibulum. Quisque ultrices pellentesque pellentesque. Pellentesque egestas quam sed blandit tempus. Sed congue nec risus posuere euismod. Maecenas ut lacus id mauris sagittis egestas a eu dui. Class aptent taciti sociosqu ad litora torquent per conubia nostra, per inceptos himenaeos. Pellentesque at ultrices tellus. Ut eu purus eget sem iaculis ultricies sed non lorem. Curabitur gravida dui eget ex vestibulum venenatis. Phasellus gravida tellus velit, non eleifend justo lobortis eget. 


\section{Cras in aliquam quam, et}
\label{sec:456}
Lorem ipsum dolor sit amet, consectetur adipiscing elit. Donec sed nunc orci. Aliquam nec nisl vitae sapien pulvinar dictum quis non urna. Suspendisse at dui a erat aliquam vestibulum. Quisque ultrices pellentesque pellentesque. Pellentesque egestas quam sed blandit tempus. Sed congue nec risus posuere euismod. Maecenas ut lacus id mauris sagittis egestas a eu dui. Class aptent taciti sociosqu ad litora torquent per conubia nostra, per inceptos himenaeos. Pellentesque at ultrices tellus. Ut eu purus eget sem iaculis ultricies sed non lorem. Curabitur gravida dui eget ex vestibulum venenatis. Phasellus gravida tellus velit, non eleifend justo lobortis eget.


\subsection{Sed pulvinar placerat enim, a}
\label{sec:00456}
Lorem ipsum dolor sit amet, consectetur adipiscing elit. Donec sed nunc orci. Aliquam nec nisl vitae sapien pulvinar dictum quis non urna. Suspendisse at dui a erat aliquam vestibulum. Quisque ultrices pellentesque pellentesque. Pellentesque egestas quam sed blandit tempus. Sed congue nec risus posuere euismod. Maecenas ut lacus id mauris sagittis egestas a eu dui. Class aptent taciti sociosqu ad litora torquent per conubia nostra, per inceptos himenaeos. Pellentesque at ultrices tellus. Ut eu purus eget sem iaculis ultricies sed non lorem. Curabitur gravida dui eget ex vestibulum venenatis. Phasellus gravida tellus velit, non eleifend justo lobortis eget.


\section{Vivamus hendrerit imperdiet ex. Vivamus}
\label{sec:123}
Lorem ipsum dolor sit amet, consectetur adipiscing elit. Donec sed nunc orci. Aliquam nec nisl vitae sapien pulvinar dictum quis non urna. Suspendisse at dui a erat aliquam vestibulum. Quisque ultrices pellentesque pellentesque. Pellentesque egestas quam sed blandit tempus. Sed congue nec risus posuere euismod. Maecenas ut lacus id mauris sagittis egestas a eu dui. Class aptent taciti sociosqu ad litora torquent per conubia nostra, per inceptos himenaeos. Pellentesque at ultrices tellus. Ut eu purus eget sem iaculis ultricies sed non lorem. Curabitur gravida dui eget ex vestibulum venenatis. Phasellus gravida tellus velit, non eleifend justo lobortis eget.



      \chapter{Conclusioni}
\label{cha:conclusioni}
Lorem ipsum dolor sit amet, consectetur adipiscing elit. Donec sed nunc orci. Aliquam nec nisl vitae sapien pulvinar dictum quis non urna. Suspendisse at dui a erat aliquam vestibulum. Quisque ultrices pellentesque pellentesque. Pellentesque egestas quam sed blandit tempus. Sed congue nec risus posuere euismod. Maecenas ut lacus id mauris sagittis egestas a eu dui. Class aptent taciti sociosqu ad litora torquent per conubia nostra, per inceptos himenaeos. Pellentesque at ultrices tellus. Ut eu purus eget sem iaculis ultricies sed non lorem. Curabitur gravida dui eget ex vestibulum venenatis. Phasellus gravida tellus velit, non eleifend justo lobortis eget. 


      %\input{capitolo4}
      
      
    \endgroup


    % bibliografia in formato bibtex
    %
    % aggiunta del capitolo nell'indice
    \addcontentsline{toc}{chapter}{Bibliografia}
    % stile con ordinamento alfabetico in funzione degli autori
    \bibliographystyle{plain}
    \bibliography{biblio}
%%%%%%%%%%%%%%%%%%%%%%%%%%%%%%%%%%%%%%%%%%%%%%%%%%%%%%%%%%%%%%%%%%%%%%%%%%
%%%%%%%%%%%%%%%%%%%%%%%%%%%%%%%%%%%%%%%%%%%%%%%%%%%%%%%%%%%%%%%%%%%%%%%%%%
%% Nota
%%%%%%%%%%%%%%%%%%%%%%%%%%%%%%%%%%%%%%%%%%%%%%%%%%%%%%%%%%%%%%%%%%%%%%%%%%
%% Nella bibliografia devono essere riportati tutte le fonti consultate 
%% per lo svolgimento della tesi. La bibliografia deve essere redatta 
%% in ordine alfabetico sul cognome del primo autore. 
%% 
%% La forma della citazione bibliografica va inserita secondo la fonte utilizzata:
%% 
%% LIBRI
%% Cognome e iniziale del nome autore/autori, la data di edizione, titolo, casa editrice, eventuale numero dell’edizione. 
%% 
%% ARTICOLI DI RIVISTA
%% Cognome e iniziale del nome autore/autori, titolo articolo, titolo rivista, volume, numero, numero di pagine.
%% 
%% ARTICOLI DI CONFERENZA
%% Cognome e iniziale del nome autore/autori (anno), titolo articolo, titolo conferenza, luogo della conferenza (città e paese), date della conferenza, numero di pagine. 
%% 
%% SITOGRAFIA
%% La sitografia contiene un elenco di indirizzi Web consultati e disposti in ordine alfabetico. 
%% E’ necessario:
%%   Copiare la URL (l’indirizzo web) specifica della pagina consultata
%%   Se disponibile, indicare il cognome e nome dell’autore, il titolo ed eventuale sottotitolo del testo
%%   Se disponibile, inserire la data di ultima consultazione della risorsa (gg/mm/aaaa).    
%%%%%%%%%%%%%%%%%%%%%%%%%%%%%%%%%%%%%%%%%%%%%%%%%%%%%%%%%%%%%%%%%%%%%%%%%%
%%%%%%%%%%%%%%%%%%%%%%%%%%%%%%%%%%%%%%%%%%%%%%%%%%%%%%%%%%%%%%%%%%%%%%%%%%
    

    \titleformat{\chapter}
        {\normalfont\Huge\bfseries}{Allegato \thechapter}{1em}{}
    % sezione Allegati - opzionale
    \appendix
    \chapter{Titolo primo allegato}

Lorem ipsum dolor sit amet, consectetur adipiscing elit. Donec sed nunc orci. Aliquam nec nisl vitae sapien pulvinar dictum quis non urna. Suspendisse at dui a erat aliquam vestibulum. Quisque ultrices pellentesque pellentesque. Pellentesque egestas quam sed blandit tempus. Sed congue nec risus posuere euismod. Maecenas ut lacus id mauris sagittis egestas a eu dui. Class aptent taciti sociosqu ad litora torquent per conubia nostra, per inceptos himenaeos. Pellentesque at ultrices tellus. Ut eu purus eget sem iaculis ultricies sed non lorem. Curabitur gravida dui eget ex vestibulum venenatis. Phasellus gravida tellus velit, non eleifend justo lobortis eget. 

\section{Titolo}
Lorem ipsum dolor sit amet, consectetur adipiscing elit. Donec sed nunc orci. Aliquam nec nisl vitae sapien pulvinar dictum quis non urna. Suspendisse at dui a erat aliquam vestibulum. Quisque ultrices pellentesque pellentesque. Pellentesque egestas quam sed blandit tempus. Sed congue nec risus posuere euismod. Maecenas ut lacus id mauris sagittis egestas a eu dui. Class aptent taciti sociosqu ad litora torquent per conubia nostra, per inceptos himenaeos. Pellentesque at ultrices tellus. Ut eu purus eget sem iaculis ultricies sed non lorem. Curabitur gravida dui eget ex vestibulum venenatis. Phasellus gravida tellus velit, non eleifend justo lobortis eget. 

\subsection{Sottotitolo}
Lorem ipsum dolor sit amet, consectetur adipiscing elit. Donec sed nunc orci. Aliquam nec nisl vitae sapien pulvinar dictum quis non urna. Suspendisse at dui a erat aliquam vestibulum. Quisque ultrices pellentesque pellentesque. Pellentesque egestas quam sed blandit tempus. Sed congue nec risus posuere euismod. Maecenas ut lacus id mauris sagittis egestas a eu dui. Class aptent taciti sociosqu ad litora torquent per conubia nostra, per inceptos himenaeos. Pellentesque at ultrices tellus. Ut eu purus eget sem iaculis ultricies sed non lorem. Curabitur gravida dui eget ex vestibulum venenatis. Phasellus gravida tellus velit, non eleifend justo lobortis eget. 


\chapter{Titolo secondo allegato}

Lorem ipsum dolor sit amet, consectetur adipiscing elit. Donec sed nunc orci. Aliquam nec nisl vitae sapien pulvinar dictum quis non urna. Suspendisse at dui a erat aliquam vestibulum. Quisque ultrices pellentesque pellentesque. Pellentesque egestas quam sed blandit tempus. Sed congue nec risus posuere euismod. Maecenas ut lacus id mauris sagittis egestas a eu dui. Class aptent taciti sociosqu ad litora torquent per conubia nostra, per inceptos himenaeos. Pellentesque at ultrices tellus. Ut eu purus eget sem iaculis ultricies sed non lorem. Curabitur gravida dui eget ex vestibulum venenatis. Phasellus gravida tellus velit, non eleifend justo lobortis eget. 

\section{Titolo}
Lorem ipsum dolor sit amet, consectetur adipiscing elit. Donec sed nunc orci. Aliquam nec nisl vitae sapien pulvinar dictum quis non urna. Suspendisse at dui a erat aliquam vestibulum. Quisque ultrices pellentesque pellentesque. Pellentesque egestas quam sed blandit tempus. Sed congue nec risus posuere euismod. Maecenas ut lacus id mauris sagittis egestas a eu dui. Class aptent taciti sociosqu ad litora torquent per conubia nostra, per inceptos himenaeos. Pellentesque at ultrices tellus. Ut eu purus eget sem iaculis ultricies sed non lorem. Curabitur gravida dui eget ex vestibulum venenatis. Phasellus gravida tellus velit, non eleifend justo lobortis eget. 

\subsection{Sottotitolo}
Lorem ipsum dolor sit amet, consectetur adipiscing elit. Donec sed nunc orci. Aliquam nec nisl vitae sapien pulvinar dictum quis non urna. Suspendisse at dui a erat aliquam vestibulum. Quisque ultrices pellentesque pellentesque. Pellentesque egestas quam sed blandit tempus. Sed congue nec risus posuere euismod. Maecenas ut lacus id mauris sagittis egestas a eu dui. Class aptent taciti sociosqu ad litora torquent per conubia nostra, per inceptos himenaeos. Pellentesque at ultrices tellus. Ut eu purus eget sem iaculis ultricies sed non lorem. Curabitur gravida dui eget ex vestibulum venenatis. Phasellus gravida tellus velit, non eleifend justo lobortis eget. 




\end{document}

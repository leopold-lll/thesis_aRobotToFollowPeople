%%%%%%%%%%%%%%%%%%%%%%%%%%%%%%%%%%%%%%%%%%%%%%%%%%%%%%%%%%%%%%%%%%%%%%
%%%%%%%%%%%%%%%%%%%%%%%%%%%%%%%%%%%%%%%%%%%%%%%%%%%%%%%%%%%%%%%%%%%%%%
% Nota:
%%%%%%%%%%%%%%%%%%%%%%%%%%%%%%%%%%%%%%%%%%%%%%%%%%%%%%%%%%%%%%%%%%%%%%
%Sommario è un breve riassunto del lavoro svolto dove si descrive l'obiettivo, l'oggetto della tesi, le metodologie e le tecniche usate, i dati elaborati e la spiegazione delle conclusioni alle quali siete arrivati.  
%
%Il sommario dell’elaborato consiste al massimo di 3 pagine e deve contenere le seguenti informazioni:
%\begin{itemize}
%	\item contesto e motivazioni 
%	\item breve riassunto del problema affrontato
%	\item tecniche utilizzate e/o sviluppate
%	\item risultati raggiunti, sottolineando il contributo personale del laureando/a
%\end{itemize}
%%%%%%%%%%%%%%%%%%%%%%%%%%%%%%%%%%%%%%%%%%%%%%%%%%%%%%%%%%%%%%%%%%%%%%
%%%%%%%%%%%%%%%%%%%%%%%%%%%%%%%%%%%%%%%%%%%%%%%%%%%%%%%%%%%%%%%%%%%%%%
\chapter*{Summary}\label{cha:summary}
\addcontentsline{toc}{chapter}{Summary}
This final dissertation is dedicated to the master thesis research project\cite{projectSourceCode} of \textbf{Stefano Leonardi} at the \textbf{University of Trento} for the \textbf{Dolomiti Robotics group}\cite{dolomitiRobotics}. The goal of this project is to implement a software able to drive a robot into a real environment.\\
\\
The scenarios that we are dealing with is a situation in which both robots and humans work. This interaction is extremely complex because the two parts need to cooperate efficiently avoiding accidents. The robot interacts with the environment with two sensors. First, the LIDAR instrument that will guarantee the driving safeness. Second, an RGB camera is used for this project to implement a new functionality: \textit{the follow procedure}. While a human is walking the robot should in real-time recognise him as the main subject, the leader, which it must follow across time through the real environment. The procedure should manage conditions in which the leader is completely hidden or it is out of the field of view for a long period of time.\\
\\
The software that will be presented has a modular structure composed of blocks. This architecture is used in order to be adaptable to future replacements of the technologies used. The general tracking problem is solved with three modules:
\begin{itemize}
	\item \textbf{Object detection} is used to periodically localize bounding boxes of people inside a frame. We have used the two actual states of the art algorithms: SSD\cite{ssd} and YOLOv3\cite{yoloV3}.
	\item \textbf{People recognition} is used to understand if the people found correspond to the leader or not. We have ideated a method based on the KNN algorithm that uses images converted into points with the DNN image classifiers: ResNet50\cite{resNet_paper} and GoogLeNet\cite{googLeNet_paper}.
	\item \textbf{Object tracking} is periodically started on the leader location and it will track it for few frames avoiding the problems generated due to long-term sequences. The methods found to be the best are: KCF\cite{kcf}, CSRT\cite{csrt} and MOSSE\cite{mosse}. Eventually, in future improvements GOTURN\cite{goturn}.
\end{itemize}
We have tested the implemented solution with an ad hoc dataset created by manually moving the robot around to recorded videos.\\
The experiments show that the software is able to be executed in real-time on low powerful processing units. In fact, robots cannot carry top quality hardware due to the high cost of the GPUs and also because these components have a too high power consumption.\\
The known weakness of the algorithm relies on the reliability of the people recognition module. In fact false-negative predictions\footnote{A false-negative prediction occurs when the leader is classified as a random person.} will affect the KNN potentialities and cause bigger and bigger problems as long as the algorithm is executed. Despite this, the procedure is able to manage minute-long frame sequences.








